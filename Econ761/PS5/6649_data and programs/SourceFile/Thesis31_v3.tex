
\documentclass[notitlepage,onecolumn,11pt]{article}
%%%%%%%%%%%%%%%%%%%%%%%%%%%%%%%%%%%%%%%%%%%%%%%%%%%%%%%%%%%%%%%%%%%%%%%%%%%%%%%%%%%%%%%%%%%%%%%%%%%%%%%%%%%%%%%%%%%%%%%%%%%%%%%%%%%%%%%%%%%%%%%%%%%%%%%%%%%%%%%%%%%%%%%%%%%%%%%%%%%%%%%%%%%%%%%%%%%%%%%%%%%%%%%%%%%%%%%%%%%%%%%%%%%%%%%%%%%%%%%%%%%%%%%%%%%%
\usepackage[page]{appendix}
\usepackage{amssymb}
\usepackage{graphicx}
\usepackage{amsmath}

\setcounter{MaxMatrixCols}{10}
%TCIDATA{OutputFilter=LATEX.DLL}
%TCIDATA{Version=5.00.0.2570}
%TCIDATA{<META NAME="SaveForMode" CONTENT="1">}
%TCIDATA{Created=Sun Dec 14 17:51:08 2003}
%TCIDATA{LastRevised=Thursday, May 08, 2008 10:18:40}
%TCIDATA{<META NAME="GraphicsSave" CONTENT="32">}
%TCIDATA{<META NAME="DocumentShell" CONTENT="General\Blank Document">}
%TCIDATA{Language=American English}
%TCIDATA{CSTFile=LaTeX article (bright).cst}

\setlength{\textheight}{8.5in}
\setlength{\topmargin}{-0.1in}
\setlength{\oddsidemargin}{0.0in}
\setlength{\evensidemargin}{0.0in}
\setlength{\textwidth}{6.2in}
\renewcommand{\baselinestretch}{1.3}
\newtheorem{theorem}{Theorem}
\newtheorem{lemma}{Lemma}
\newtheorem{corollary}{Corollary}
\newtheorem{proposition}{Proposition}
\newtheorem{acknowledgement}[theorem]{Acknowledgement}
\newtheorem{algorithm}[theorem]{Algorithm}
\newtheorem{axiom}[theorem]{Axiom}
\newtheorem{case}[theorem]{Case}
\newtheorem{claim}[theorem]{Claim}
\newtheorem{conclusion}[theorem]{Conclusion}
\newtheorem{condition}[theorem]{Condition}
\newtheorem{criterion}[theorem]{Criterion}
\newtheorem{definition}[theorem]{Definition}
\newtheorem{example}[theorem]{Example}
\newtheorem{exercise}[theorem]{Exercise}
\newtheorem{notation}[theorem]{Notation}
\newtheorem{problem}[theorem]{Problem}
\newtheorem{remark}[theorem]{Remark}
\newtheorem{solution}[theorem]{Solution}
\newtheorem{summary}[theorem]{Summary}
\newenvironment{proof}[1][Proof]{\textbf{#1.} }{\ \rule{0.5em}{0.5em}}
\input{tcilatex}
\pagenumbering{roman}
\renewcommand{\setthesection}{\Alph{section}}

\begin{document}

\title{What Happens When Wal-Mart Comes to Town: An Empirical Analysis of
the Discount Retailing Industry\thanks{%
This paper is a revision of chapter one of my thesis. I am deeply indebted
to my committee members -- Steven Berry, Penny Goldberg, Hanming Fang, and
Philip Haile -- for their continual support and encouragement. Special
thanks go to Pat Bayer and Alvin Klevorick, who have been very generous with
their help. I also thank the editor, three anonymous referees, Donald
Andrews, Pat Bajari, Donald Brown, Judy Chevalier, Tom\ Holmes, Yuichi
Kitamura, Ariel Pakes, Herbert Scarf, and seminar participants at Boston
University, Columbia University, Duke University, Harvard, MIT, Northwestern
University, NYU, Princeton,\ Stanford University, UCLA, UCSD, University of
Chicago, University of Michigan, University of Minnesota, University of\
Pennsylvania, and participants of the 2006 \textit{Review of Economic Studies%
} European Meetings in Oslo, Essex, and Tel Aviv for many helpful comments.
Financial support from Cowles Foundation Prize and Horowitz Foundation
Fellowship is gratefully acknowledged. All errors are my own.}}
\author{Panle Jia\thanks{%
Department of Economics, M.I.T. and NBER. Email: pjia@mit.edu. Comments are
welcome.}}
\date{First version: November 2005. This version: January 2008}
\maketitle

\begin{abstract}
In the past few decades multi-store retailers, especially those with a
hundred or more stores, have experienced substantial growth. At the same
time, there is widely reported public outcry over the impact of these chain
stores on other retailers and local communities. This paper develops an
empirical model to assess the impact of chain stores on other discount
retailers and to quantify the size of the scale economies within a chain.
The model has two key features. First, it allows for flexible competition
patterns among all players. Second, for chains, it incorporates the scale
economies that arise from operating multiple stores in nearby regions. In
doing so, the model relaxes the commonly used assumption that entry in
different markets is independent. The paper exploits the lattice theory to
solve this complicated entry game among chains and other discount retailers
in a large number of markets. The paper finds that the negative impact of
Kmart's presence on Wal-Mart's profit is much stronger in 1988 than in 1997,
while the opposite is true for the effect of Wal-Mart's presence on Kmart's
profit. Having a chain store in a market makes roughly fifty percent of the
discount stores unprofitable. Wal-Mart's expansion from the late 1980s to
the late 1990s explains about forty to fifty percent of the net change in
the number of small discount stores, and thirty to forty percent for all
other discount stores. Scale economies were important for Wal-Mart, but less
so for Kmart, and the magnitude did not grow proportionately with the
chains' sizes.

Keywords: Chain, Entry, Spatial Correlation, Wal-Mart, Lattice\vspace{0.12in}

\thispagestyle{empty}\pagebreak
\end{abstract}

\pagenumbering{arabic}\textquotedblleft Bowman's [in a small town in
Georgia] is the eighth \textquotedblleft main street\textquotedblright\
business to close since Wal-Mart came to town.\ldots\ For the first time in
seventy-three years the big corner store is empty.\textquotedblright\ Archer
and Taylor, \textit{Up against the Wal-Mart}.

\begin{quote}
\textquotedblleft There is ample evidence that a small business need not
fail in the face of competition from large discount stores. In fact, the
presence of a large discount store usually acts as a magnet, keeping local
shoppers\ldots .and expanding the market\ldots .\textquotedblright\ Morrison
Cain, Vice president of International Mass Retail Association.
\end{quote}

\section{Introduction}

The landscape of the U.S. retail industry has changed considerably over the
past few decades as the result of two closely related trends. One is the
rise of discount retailing; the other is the increasing prevalence of large
retail chains. In fact, the discount retailing sector is almost entirely
controlled by chains. In 1997, the top three chains (Wal-Mart, Kmart, and
Target) accounted for 72.7\% of total sector sales and 54.3\% of the
discount stores.

Discount retailing is a fairly new concept, with the first discount stores
appearing in the 1950s. The leading magazine for the discount industry, 
\textit{Discount Merchandiser}, defines a modern discount store as a
departmentalized retail establishment that makes use of self-service
techniques to sell a large variety of hard goods and soft goods at uniquely
low margins.\footnote{%
See the annual report \textquotedblleft The True Look of the Discount
Industry\textquotedblright\ in the June issue of \textit{Discount
Merchandiser} for the definition of the discount retailing, the sales and
store numbers for the top 30 largest firms, as well as the industry sales
and total number of discount stores.}$^{,}$\footnote{%
According to \textit{Annual Benchmark Report for Retail Trade and Food
Services} published by the Census Bureau, from 1993 to 1997, the average
markup for regular department stores was 27.9\%, while the average markup
for discount stores was 20.9\%. Both markups increased slightly from 1998 to
2000.} Over the span of several decades, the sector has emerged from the
fringe of the retail industry and become one of the major sectors of the
retail industry.\footnote{%
The other retail sectors are:\ building materials, food stores, automotive
dealers, apparel, furniture, eating and drinking places, and miscellaneous
retail.} From 1960 to 1997, the total sales revenue of discount stores, in
real terms, increased 15.6 times, compared with an increase of 2.6 times for
the entire retail industry.

As the discount retailing sector continues to grow, opposition from other
retailers, especially small ones, begins to mount. The critics tend to
associate discounters and other big retailers with small-town problems
caused by the closing of small firms, such as the decline of downtown
shopping districts, eroded tax bases, decreased employment, and the
disintegration of closely knit communities. Partly because tax money is used
to restore the blighted downtown business districts and to lure the business
of big retailers with various forms of economic development subsidies, the
effect of big retailers on small firms and local communities has become a
matter of public concern.\footnote{%
See \textit{The Shils Report (1997): Measuring the Economic and Sociological
Impact of the Mega-Retail Discount Chains on Small Enterprises in Urban,
Suburban and Rural Communities}.} My first goal in this paper is to quantify
the impact of national discount chains on the profitability and entry and
exit decisions of small retailers from the late 1980s to the late 1990s.

The second salient feature of retail development in the past several
decades, including in the discount sector, is the increasing dominance of
large chains. In 1997, retail chains with a hundred or more stores accounted
for 0.07\% of the total number of firms, yet they controlled 21\% of the
establishments and accounted for 37\% of sales and 46\% of retail employment.%
\footnote{%
See the 1997 Economic Census Retail Trade subject series \textit{%
Establishment and Firm Size (Including Legal Form of Organization)},
published by the US Census Bureau.} Since the late 1960s, their share of the
retail market more than doubled. In spite of the dominance of chain stores,
few empirical studies (except Holmes (2005) and Smith (2004)) have
quantified the potential advantages of chains over single-unit firms, in
part because of the modeling difficulties.\footnote{%
I discuss Holmes (2005) in detail below. Smith (2004) estimates the demand
cross-elasticities between stores of the same firm and finds that mergers
between the largest retail chains would increase the price level by up to
7.4\%.} In entry models, for example, the store entry decisions of
multi-unit chains are related across markets. Most of the literature assumes
that entry decisions are independent across markets and focuses on
competition among firms within each local market. My second objective here
is to extend the entry literature by relaxing the independence assumption,
and to quantify the advantage of operating multiple units by explicitly
modeling chains' entry decisions in a large number of markets.

The model has two key features. First, it allows for flexible competition
patterns among all retailers. Second, it incorporates the potential benefits
of locating multiple stores near one another. Such benefits, which I group
as \textquotedblleft the chain effect,\textquotedblright\ can arise through
several different channels. For example, there may be significant scale
economies in the distribution system. Stores located near each other can
split advertising costs or employee training costs, or they can share
knowledge about the specific features of local markets.

The chain effect causes profits of stores in the same chain to be spatially
related. As a result, choosing store locations to maximize total profit is
complicated, since with $N$ markets there are $2^{N}$ possible location
choices. In the current application, there are more than 2,000 markets and
the number of possible location choices exceeds 10$^{600}.$ When several
chains compete against each other, solving for the Nash equilibrium becomes
further involved, as firms balance the gains from the chain effect against
competition from rivals. I tackle this problem in several steps. First, I
transform the profit maximization problem into a search for the fixed points
of the necessary conditions. This transformation shifts the focus of the
problem from a set with $2^{N}$ elements to the set of fixed points of the
necessary conditions. The latter has a much smaller dimension, and is
well-behaved with easy-to-locate minimum and maximum points. Having dealt
with the problem of dimensionality, I take advantage of the supermodularity
property of the game to search for the Nash equilibrium. Finally, in
estimating the parameters, I adopt the econometric technique proposed by
Conley (1999) to address the issue of cross-sectional dependence.

The algorithm proposed above exploits the game's supermodularity structure
to solve a complicated problem. It has a couple of limitations. First, it is
not applicable to oligopoly games with three or more chains.\footnote{%
Entry games are not supermodular in general as the competition effect is
usually assumed to be negative. However, with only two chains, we can
redefine the strategy space for one player to be the negative of the
original space. Then the game associated with the new strategy space is
supermodular, provided that each chain's profit function is supermodular in
its own strategy. See section 5.2 for details. This would not work for
oligopoly games with three or more chains.} The algorithm is also not
applicable to situations where stores from the same chain compete for sales
and the net chain effect becomes negative, as the business stealing effect
overwhelms the positive spill-over effect.\footnote{%
See section 6.2.3 for further discussions.} Extending the algorithm to
address both issues is left for future research.

The analysis exploits a unique data set I collected that covers the entire
discount retailing industry from 1988 to 1997, during which the two major
national chains were Kmart and Wal-Mart.\footnote{%
During the sample period, Target was a regional store that competed mostly
in the big metropolitan areas in the Midwest with few stores in the sample.
See the data section for more details.} The results indicate that the
negative impact of Kmart's presence on Wal-Mart's profit is much stronger in
1988 than in 1997, while the opposite is true for the effect of Wal-Mart's
presence on Kmart's profit. Having a chain store in a market makes roughly
fifty percent of the discount stores unprofitable. Wal-Mart's expansion from
the late 1980s to the late 1990s explains about 37\% to 55\% percent of the
net change in the number of small discount stores, and 34\% to 41\% for all
other discount stores. Scale economies were important to Wal-Mart, but less
so for Kmart, and their importance did not grow proportionately with the
size of the chains. Finally, government subsidies to either chains or small
firms in this industry are not likely to be effective in increasing the
number of firms or the level of employment.

The paper complements a recent study by Holmes (2005), which analyzes the
diffusion process of Wal-Mart stores. Holmes' approach is appealing because
he derives the magnitude of the economies of density, a concept similar to
the chain effect in this paper, from the dynamic expansion process. In
contrast, I identify the chain effect from the stores' geographic clustering
pattern. My approach abstracts from a number of important dynamic
considerations. For example, it does not allow firms to delay store openings
because of credit constraints, nor does it allow for any preemption motive
as the chains compete and make simultaneous entry decisions. A dynamic model
that incorporates both the competition effects and the chain effect would be
ideal.\ However, given the great difficulty of estimating the economies of
density in a single agent dynamic model, as Holmes (2005) shows, it is
currently infeasible to estimate a dynamic model that also incorporates the
strategic interactions within chains and between chains and small retailers.
Since one of my main goals is to analyze the competition effects and perform
policy evaluations, I adopt a three-stage model. In the first stage, or the
\textquotedblleft pre-chain\textquotedblright\ period, small retailers make
entry decisions without anticipating the future entry of Kmart or Wal-Mart.
In the second stage, Kmart or Wal-Mart emerge in the retail industry and
optimally locate their stores across the entire set of markets. In the third
stage, existing small firms decide whether to continue their business, while
potential entrants decide whether to enter the market and compete with the
chains. The extension of the current framework to a dynamic model is left
for future research.

This paper contributes to the entry literature initiated by Bresnahan and
Reiss (1990, 1991) and Berry (1992), where researchers infer the firms'
underlying profit functions by observing their equilibrium entry decisions
across a large number of markets. To the extent that retail chains can be
treated as multi-product firms whose differentiated products are stores with
different locations, this paper relates to several recent empirical entry
papers that endogenize firms' product choices upon entry. For example,
Mazzeo (2002) considers the quality choices of highway motels, and Seim
(2005) studies how video stores soften competition by choosing different
locations. Unlike these studies, in which each firm chooses only one
product, I analyze the behavior of multi-product firms whose product spaces
are potentially large.

This paper is also related to a large literature on spatial competition in
retail markets, for example, Pinkse \textit{et. al. }(2002), Smith (2004),
and Davis (2006). All of these models take the firms' locations as given and
focus on price or quantity competition. I adopt the opposite approach.
Specifically, I assume a parametric form for the firms' reduced-form profit
functions from the stage competition, and examine how they compete spatially
by balancing the chain effect against the competition effect of rivals'
actions on their own profits.

Like many other discrete-choice models with complete information, the entry
games generally allow multiple equilibria. There is a very active literature
on estimating discrete-choice games that explicitly addresses the issue of
multiple equilibria. For example, Tamer (2003) proposes an exclusion
condition that leads to point identification in two-by-two games. Andrews,
Berry and Jia (2004), Chernozhukov, Hong, and Tamer (2007), Pakes, Porter,
Ho, and Ishii (2005), Romano and Shaikh (2006) etc., analyze bound
estimations that exploit inequality constraints derived from necessary
conditions. Bajari, Hong, and Ryan (2007) examine the identification and
estimation of the equilibrium selection mechanism as well as the payoff
function. Ciliberto and Tamer (2006) study multiple equilibria in the
airline markets and use Chernozhukov, Hong, and Tamer (2004) to construct
the confidence region. Ackerberg and Gowrisakaran (2007) analyze banks'
adoptions of the automated clearinghouse electronic payment system, assuming
each network is in one of the two extreme equilibria with a certain
probability. In the current application, I estimate the parameters using the
equilibrium that is most profitable for Kmart, and also provide the
parameter estimates at two other different equilibria as a robustness check.

The paper's algorithm is an application of the lattice theory, in particular
Tarski (1955)'s fixed point theorem and Topkis\ (1978)'s monotonicity
theorem. Milgrom and Shannon (1994) derived a necessary and sufficient
condition for the solution set of an optimization problem to be monotonic in
the parameters of the problem. Athey (2002) extended the monotone
comparative statics to situations with uncertainty.

Finally, the paper is part of the growing literature on Wal-Mart, which
includes Stone (1995), Basker (2005a, 2005b), Hausman and Leibtag (2005),
Neumark \textit{et al} (2005), and Zhu and Singh (2007).

The remainder of the paper is structured as follows. Section 2 provides
background information about the discount retailing sector. Section 3
describes the data set, and section 4 discusses the model. Section 5
proposes a solution algorithm for the game between chains and small firms
when there is a large number of markets. Section 6 explains the estimation
approach. Section 7 presents the results. Section 8 concludes. The appendix
outlines the technical details not covered in section 5.

\section{Industry background}

Discount retailing is one of the most dynamic sectors in the retail
industry. Table I displays some statistics for the industry from 1960 to
1997. The sales revenue for this sector, in 2004 US dollars, skyrocketed
from 12.8 billion in 1960 to 198.7 billion in 1997. In comparison, the sales
revenue for the entire retail industry increased only modestly from 511.2
billion to 1313.3 billion during the same period. The number of discount
stores multiplied from 1329 to 9741, while the number of firms dropped from
1016 to 230.

Chain stores dominate the discount retailing sector, as they do other retail
sectors. In 1970, the 39 largest discount chains, with twenty-five or more
stores each, operated 49.3\% of the discount stores and accounted for 41.4\%
of total sales. By 1989, both shares had increased to roughly 88\%. In 1997,
the top 30 chains controlled about 94\% of total stores and sales.

The principal advantages of chain stores include the central purchasing
unit's ability to buy on favorable terms and to foster specialized buying
skills; the possibility of sharing operating and advertising costs among
multiple units; the freedom to experiment in one selling unit without risk
to the whole operation. Stores also frequently share their private
information about local markets and learn from one another's managerial
practices. Finally, chains can achieve economies of scale by combining
wholesaling and retailing operations within the same business unit.

Until the late 1990s, the two most important national chains were Kmart and
Wal-Mart. Each firm opened its first store in 1962. The first Kmart was
opened by the variety-chain Kresge. Kmart stores were a new experiment that
provided consumers with quality merchandise at prices considerably lower
than those of regular retail stores. To reduce advertising costs and to
minimize customer service, these stores emphasized nationally advertised
brand-name products. Consumer satisfaction was guaranteed, and all goods
could be returned for a refund or an exchange (See Vance and Scott (1994),
pp32). These practices were an instant success, and Kmart grew rapidly in
the 1970s and 1980s. By the early 1990s, the firm had more than 2200 stores
nationwide. In the late 1980s, Kmart tried to diversify and pursued various
forms of specialty retailing in pharmaceutical products, sporting goods,
office supplies, building materials, etc. The attempt was unsuccessful, and
Kmart eventually divested itself of these interests by the late 1990s.
Struggling with its management failures throughout the 1990s, Kmart
maintained roughly the same number of stores; the opening of new stores
offset the closing of existing ones.

Unlike Kmart, which was initially supported by an established retail firm,\
Wal-Mart started from scratch and grew relatively slowly in the beginning.
To avoid direct competition with other discounters, it focused on small
towns in southern states where there were few competitors. Starting in the
early 1980s, the firm began its aggressive expansion process that averaged
140 store openings per year. In 1991, Wal-Mart replaced Kmart as the largest
discounter. By 1997, Wal-Mart had 2362 stores (not including the wholesale
clubs) in all states, including Alaska and Hawaii.

As the discounters continue to grow, other retailers start to feel their
impact. There are extensive media reports on the controversies associated
with the impact of large chains on small retailers and on local communities
in general. As early as 1994, the United States House of Representatives
convened a hearing titled \textquotedblleft The\ Impact of Discount
Superstores on Small Businesses and Local Communities.\textquotedblright\
Witnesses from mass retail associations and small retail councils testified,
but no legislation followed, partly due to a lack of concrete evidence. In
April 2004, the University of California, Santa Barbara, held a conference
that centered on the cultural and social impact of the leading discounter,
Wal-Mart. In November 2004, both CNBC and PBS aired documentaries that
displayed the changes Wal-Mart had brought to the society.

\section{Data}

The available data sets dictate the modeling approach used in this paper.
Hence, I discuss them before introducing the model.

\subsection{Data sources}

There are three main data sources. The data on discount chains come from an
annual directory published by Chain Store Guide Inc. The directory covers
all operating discount stores of more than ten thousand square feet. For
each store, the directory lists its name, size, street address, telephone
number, store format, and firm affiliation.\footnote{%
The directory stopped providing store size information in 1997 and changed
the inclusion criterion to 20,000 square feet in 1998. The store formats
include membership stores, regional offices, and in later years distribution
centers.} The U.S. industry classification system changed from the Standard
Industrial Classification System (SIC) to the North American Industry
Classification System (NAICS) in 1998. To avoid potential inconsistencies in
the industry definition, I restrict the sample period to the ten years
before the classification change. As first documented in Basker (2005), the
directory was not fully updated for some years. Fortunately, it was fairly
accurate for the years used in this study. See appendix A for details.

The second data set, the County Business Patterns (CBP), tabulates at the
county level the number of establishments by employment size category by
industry sectors.\footnote{%
CBP reports data at the establishment level, not the firm level. As it does
not include information on firm ownership, I do not know which
establishments are owned by the same firm. Given this data limitation, I
have assumed that, in contrast to chain stores, all small retailers are
single-unit firms. Throughout this paper, the terms \textquotedblleft small
firms\textquotedblright\ and \textquotedblleft small
stores\textquotedblright\ will be used interchangeably.} There are eight
retail sectors at the two-digit SIC level: building materials and garden
supplies, general merchandise stores (or discount stores), food stores,
automotive dealers and service stations, apparel and accessory stores,
furniture and home-furnishing stores, eating and drinking places, and
miscellaneous retail. Both Kmart and Wal-Mart are classified as a firm in
the general merchandise sector. I focus on two groups of retailers that
compete with them: a) small general merchandise stores with nineteen or
fewer employees; b) all retailers in the general merchandise sector. I\ also
experimented unsuccessfully with modeling the competition between these
chains and retailers in a group of sectors. The model is too stylized to
accommodate the vast differences between retailers in different sectors.

The number of retailers in the \textquotedblleft
pre-chain\textquotedblright\ period comes from 1978's CBP data. Data prior
to 1977 are in the tape format and not readily usable. I downloaded the
county business pattern data from the Geospatial and Statistical Data Center
of University of Virginia.\footnote{%
The web address is (as of January 2008):
http://fisher.lib.virginia.edu/collections/stats/ccdb/.}

Data on county level population are downloaded from the websites of U.S.
Census Bureau (before 1990) and the Missouri State Census Data Center (after
1990). Other county level demographic and retail sales data are from various
years of the decennial census and the economic census.

\subsection{Market definition and data description\label{Market}}

In this paper, a market is defined as a county. Although the Chain Store
Guide publishes the detailed street addresses for the discount stores,
information about small firms is available only at the county level. Many of
the market size variables, like retail sales, are also available only at the
county level.

I focus on counties with an average population between 5,000 and 64,000 from
1988 to 1997. There are 2065 such counties among a total of 3140 in the U.S.
According to Vance and Scott (1994), the minimum county population for a
Wal-Mart store was 5,000 in the 1980s, while Kmart concentrated in places
with a much larger population. 9\% of all U.S. counties were smaller than
5,000 and were unlikely to be a potential market for either chain, while
25\% of them were large metropolitan areas with an average population of
64,000 or more. These big counties typically included multiple
self-contained shopping areas, and consumers were unlikely to travel across
the entire county to shop. The market configuration in these big counties
was very complex with a large number of competitors and many market niches.
Defining a county as a market is likely to be problematic for these
counties. Given the data limitation, I model entry decisions in those 2065
small- to medium-sized counties, and treat the chain stores in the other
counties as exogenously given. The limitation of this approach is that the
spillover effect from the chain stores located in large counties is also
treated as exogenous. Using a county as a market definition also assumes
away the cross-border shopping behavior. In future research, any data on the
geographic patterns of consumers' shopping behavior would enable a more
reasonable market definition.

During the sample period, there were two national chains:\ Kmart and
Wal-Mart. The third largest chain, Target, had 340 stores in 1988 and about
800 stores in 1997. Most of them were located in metropolitan areas in the
Midwest, with on average fewer than twenty stores in the counties studied
here. I do not include Target in the analysis.

In the sample, only eight counties had two Kmart stores and forty-nine
counties had two Wal-Mart stores in 1988; the figures were eight and
sixty-six counties, respectively, in 1997. The current specification
abstracts from the choice of the number of opening stores and considers only
market entry decisions, as there is not enough variation in the data to
identify the profit of the second store in the same market. In section \ref%
{Extension}, I discuss how to extend the algorithm to allow for multiple
stores in any given market.

Table II presents summary statistics of the sample for 1978, 1988, and 1997.
The average county population was 21,470 in 1978. It increased by 5\%
between 1978 and 1988, and 8\% between 1988 and 1997. Retail sales per
capita, in 1984 dollars, was \$4,070 in 1977. It dropped to \$3,690 in 1988,
but recovered to \$4,050 in 1997. The average percentage of urban population
was 30\% in 1978. It barely changed between 1978 and 1988, but increased to
33\% in 1997. About one quarter of the counties was primarily rural with a
small urban population, which is why the average across the counties seems
somewhat low. 41\% of the counties were in the Midwest (which includes the
Great Lakes region, the Plains region, and the Rocky Mountain region, as
defined by the Bureau of Economic Analysis), and 50\% of the counties were
in the southern region (which includes the Southeast region and the
Southwest region), with the rest in the Far West and the Northeast regions.
Kmart had stores in 21\% of the counties in 1988. The number dropped
slightly to 19\% in 1997. In comparison, Wal-Mart had stores in 32\% of the
counties in 1988 and 48\% in 1997. I do not have data on the number of Kmart
and Wal-Mart stores in the sample counties in 1978. Before the 1980s, Kmart
was mainly operating in large metropolitan areas, and Wal-Mart was only a
small regional firm. I assume that in 1978, retailers in my sample counties
did not face competition from these two chains.

In 1978, the average number of discount stores per county was 4.89. The
majority of them were quite small, as stores with 1-19 employees accounted
for 4.75 of them. In 1988, the number of discount stores (excluding Kmart
and Wal-Mart stores) dropped to 4.54, while that of small stores (with 1-19
employees) dropped to 3.79. By 1997, these numbers further declined to 4.04
and 3.46, respectively.

\section{Modeling}

\subsection{Model setup\label{Model}}

The model I develop is a three-stage game. Stage one corresponds to the
\textquotedblleft pre-chain\textquotedblright\ period when only small firms
compete against each other.\footnote{%
In the empirical application, I also estimate the model using all discount
stores, not just small stores. See section 7 for details.} They enter the
market if profit after entry recovers the sunk cost. In stage two, Kmart and
Wal-Mart simultaneously choose store locations to maximize their total
profits in all markets. In the last stage, existing small firms decide
whether to continue their business, while potential entrants decide whether
to enter the market to compete with the chain stores and the existing small
stores. Small firms are single-unit stores and only enter one market. In
contrast, Kmart and Wal-Mart operate many stores and compete in multiple
markets.

This is a complete-information game except for one major difference: in the
first stage, small firms make entry decisions without anticipating Kmart
and\ Wal-Mart in the later period. The emergence of Kmart and Wal-Mart in
the second stage is an unexpected event for the small firms. Once Kmart and
Wal-Mart have appeared on the stage, all firms obtain full knowledge of
their rivals' profitability and the payoff structure. Facing a number of
existing small firms in each market, Kmart and Wal-Mart make location
choices, taking into consideration small retailers' adjustment in the third
stage. Finally, unprofitable small firms exit the market, and new entrants
come in. Once these entry decisions are made, firms compete and profits are
realized. Notice that I have implicitly assumed that chains can commit to
their entry decisions in the second stage and do not further adjust after
the third stage. This is based on the observation that most chain stores
enter with a long-term lease of the rental property, and in many cases they
invest considerably in the infrastructure construction associated with
establishing a big store.

The three-stage model is motivated by the fact that small retailers existed
long before the era of the discount chains. Accordingly, the first stage
should be considered as \textquotedblleft historical\textquotedblright\ and
the model uses this stage to fit the number of small retailers before the
entry of Kmart and Wal-Mart. Stage two and three happen roughly
concurrently: small stores adjust quickly once they observe big chains'
decisions.

\subsection{The profit function\label{ProfitFn}}

One way to obtain the profit function is to start from primitive assumptions
of supply and demand in the retail markets, and derive the profit functions
from the equilibrium conditions. Without any price, quantity, or sales data,
and with very limited information on store characteristics, this approach is
extremely demanding on data and relies heavily on the primitive assumptions.
Instead, I follow the convention in the entry literature and assume that
firms' profit functions take a linear form and that profits decline in the
presence of rivals.

In the first stage, or the \textquotedblleft pre-chain\textquotedblright\
period, profit for a small store that operates in market $m$ is:%
\begin{equation}
\Pi _{s,m}^{0}=X_{m}^{0}\beta _{s}+\delta _{ss}\ln (N_{s,m}^{0})+\sqrt{%
1-\rho ^{2}}\varepsilon _{m}^{0}+\rho \eta _{s,m}^{0}-SC  \label{Profit_0}
\end{equation}%
where \textquotedblleft $s$\textquotedblright\ stands for small stores.
Profit from staying outside the market is normalized to 0 for all players.

There are several components in the small store's profit $\Pi _{s,m}^{0}$:
the observed market size $X_{m}^{0}\beta _{s}$ that is parameterized by
demand shifters, like population, the extent of urbanization, etc.; the
competition effect $\delta _{ss}\ln (N_{s,m}^{0})$ that is monotonically
increasing (in the absolute value)\ in the number of competing stores $%
N_{s,m};$ the unobserved profit shock $\sqrt{1-\rho ^{2}}\varepsilon
_{m}^{0}+\rho \eta _{s,m}^{0}$, known to the firms but unknown to the
econometrician; and the sunk cost of entry $SC$. As will become clear below,
both the vector of observed market size variables $X_{m}$ and the
coefficients $\beta $ are allowed to vary across different players. For
example, Kmart might have some advantage in the Midwest, Wal-Mart stores
might be more profitable in markets close to their headquarters, and small
retailers might find it easier to survive in rural areas.

The unobserved profit shock has two elements: $\varepsilon _{m}^{0},$ the
market-level profit shifter that affects all firms operating in the market,
and $\eta _{s,m}^{0},$ a firm-specific profit shock. $\varepsilon _{m}^{0}$
is assumed to be i.i.d. across markets, while $\eta _{s,m}^{0}$ is assumed
to be i.i.d. across both firms and markets. $\sqrt{1-\rho ^{2}}$ (with $%
0\leq \rho \leq 1$) measures the importance of the market common shock. In
principle, its impact can differ between chains and small firms. For
example, the market specific business environment -- how developed the
infrastructure is, whether the market has sophisticated shopping facilities,
and the stance of the local community toward large corporations including
big retailers -- might matter more to chains than to small firms. In the
baseline specification, I restrict $\rho $ to be the same across all
players. Relaxing it does not improve the fit much. $\eta _{s,m}^{0}$
incorporates the unobserved store level heterogeneity, including the
management ability, the display style and shopping environment, employees'
morale or skills, etc. As is standard in discrete choice models, the scale
of the parameter coefficients and the variance of the error term are not
separately identified. I normalize the variance of the error term to $1.$ In
addition, I\ assume that both $\varepsilon _{m}^{0}$ and $\eta _{s,m\text{ }%
}^{0}$ are standard normal random variables for computational convenience.
In other applications, a more flexible distribution (like a mixture of
normals) might be more appropriate.

As mentioned above, in the \textquotedblleft pre-chain\textquotedblright\
stage, the small stores make entry decisions without anticipating Kmart's
and Wal-Mart's entry. As a result, $N_{s,m}^{0}$ is only a function of $%
(X_{m}^{0},\varepsilon _{m}^{0},\eta _{s,m}^{0}),$ and is independent of
Kmart's and Wal-Mart's entry decisions in the second stage.

To describe the choices of Kmart and Wal-Mart, let me introduce some vector
variables. Let $D_{i,m}\in \{0,1\}$ stand for chain $i$'s strategy in market 
$m$, where $D_{i,m}$ $=1$ if chain $i$ operates a store in market $m$ and $%
D_{i,m}=0$ otherwise. $D_{i}=\{D_{i,1},...,D_{i,M}\}$ is a vector indicating
chain $i$'s location choices for the entire set of markets. $D_{j,m}$ denote
rival $j$'s strategy in market $m$. Finally, let $Z_{ml}$ designate the
distance from market $m$ to market $l$ in miles, and $Z_{m}=%
\{Z_{m1},...,Z_{mM}\}.$

The following equation system describes the payoff structure during the
\textquotedblleft post-chain\textquotedblright\ period when small firms
compete against the chains.\footnote{%
I\ treat stage-two and stage-three as happening \textquotedblleft
concurrently\textquotedblright\ by assuming that both chains and small firms
share the same market-level shock $\varepsilon _{m}.$} The first equation is
the profit for chains, the second equation is the profit for small firms,
and the last equation defines how the market unobserved profit shocks evolve
over time:%
\begin{equation}
\left\{ 
\begin{array}{c}
\Pi _{i,m}(D_{i},D_{j,m},N_{s,m};~X_{m},Z_{m},\varepsilon _{m},\eta
_{i,m})=D_{i,m}\ast \lbrack X_{m}\beta _{i}+\delta _{ij}D_{j,m}+\delta
_{is}\ln (N_{s,m}+1)+ \\ 
\ \ \ \ \ \ \ \ \ \ \ \ \ \ \ \ \ \ \ \ \ \ \ \ \ \ \ \ \ \ \ \ \ \ \ \delta
_{ii}\Sigma _{l\neq m}\frac{D_{i,l}}{Z_{ml}}+\sqrt{1-\rho ^{2}}\varepsilon
_{m}+\rho \eta _{i,m}],~i,j\in \{k,w\} \\ 
\Pi _{s,m}(D_{i},D_{j,m},N_{s,m};~X_{m},\varepsilon _{m},\eta
_{i,m})=X_{m}\beta _{s}+\sum_{i=k,w}\delta _{si}D_{i,m}+\delta _{ss}\ln
(N_{s,m})+ \\ 
\ \ \ \ \ \ \ \ \ \ \ \ \ \ \ \ \ \ \ \ \ \ \ \ \sqrt{1-\rho ^{2}}%
\varepsilon _{m}+\rho \eta _{s,m}-SC\ast 1[\text{new entrant}],\rho \in
\lbrack 0,1] \\ 
\varepsilon _{m}=\tau \varepsilon _{m}^{0}+\sqrt{1-\tau ^{2}}\tilde{%
\varepsilon}_{m},\ \tau \in \lbrack 0,1]%
\end{array}%
\right.  \label{Profit}
\end{equation}%
where `$k$' denotes Kmart, `$w$' Wal-Mart, and `$s$' small firms. In the
following, I\ discuss each equation in turn.

First, notice the presence of $D_{i}$ in chain $i$'s profit $\Pi
_{i,m}(\cdot )$: profit in market $m$ depends on the number of stores chain $%
i$ has in other markets. Chains maximize their total profits in all markets $%
\Sigma _{m}\Pi _{i,m}$, internalizing the spillover effect among stores in
different locations.

As mentioned above, $X_{m}\beta _{i}$ is indexed by $i$ so that the market
size can have differential impact on firms' profitability. The competition
effect from the rival chain is captured by $\delta _{ij}D_{j,m},$ where $%
D_{j,m}$ is one if rival $j$ operates a store in market $m$. $\delta
_{is}\ln (N_{s,m}+1)$ denotes the effect of small firms on chain $i$'s
profit. The addition of 1 in $\ln (N_{s,m}+1)$ is used to avoid $\ln 0$ for
markets without any small firms. The log form allows the incremental
competition effect to taper off when there are many small firms. In
equilibrium, the number of small firms in the last stage is a function of
Kmart's and Wal-Mart's decisions: $N_{s,m}(D_{k,m},D_{w,m})$. When making
location choices, the chains take into consideration the impact of small
firms' reactions on their own profits.

The chain effect is denoted by $\delta _{ii}\Sigma _{l\neq m}\frac{D_{i,l}}{%
Z_{ml}},$ the benefit that having stores in other markets generates for the
profit in market $m.$ $\delta _{ii}$ is assumed to be non-negative. Nearby
stores split the costs of operation, delivery, and advertising to achieve
scale economies. They also share knowledge of local markets and learn from
one another's managerial success. All these factors suggest that having
stores nearby benefits the operation in market $m$, and that the benefit
declines with the distance. Following Bajari and Fox (2005), I divide the
spillover effect by the distance between the two markets $Z_{ml},$ so that
profit in market $m$ is increased by $\delta _{ii}\frac{D_{i,l}}{Z_{ml}}$ if
there is a store in market $l$ that is $Z_{ml}$ miles away. This simple
formulation serves two purposes. First, it is a parsimonious way to capture
the fact that it might be increasingly difficult to benefit from stores that
are farther away. Second, the econometric technique exploited in the
estimation requires the dependence among observations to die away
sufficiently fast. I also assume that the chain effect takes place among
counties whose centroids are within fifty miles, or roughly an area that
expands seventy-five miles in each direction.\ Including counties within a
hundred miles increases the computing time with little change in the
parameters.

This paper focuses on the chain effect that is \textquotedblleft
localized\textquotedblright\ in nature. Some chain effects are
\textquotedblleft global\textquotedblright : for example, the gain that
arises from a chain's ability to buy a large volume at a discount. The
latter benefits affect all stores the same, and cannot be separately
identified from the constant of the profit function. Hence, the estimates $%
\delta _{ii}$, should be interpreted as a lower bound to the actual
advantages enjoyed by a chain.

As in small firms' profit function, chain $i$'s profit shock contains two
elements:\ the market shifter common to all firms $\varepsilon _{m},$ and
the firm specific shock $\eta _{i,m}$. Both are assumed to have a standard
normal distribution.

Small firms' profit in the last stage $\Pi _{s,m}$ is similar to the
\textquotedblleft pre-chain\textquotedblright\ period, except for two
elements. First, $\Sigma _{i=k,w}\delta _{si}D_{i,m}$ captures the impact of
Kmart and Wal-Mart on small firms. Second, only new entrants pay the sunk
cost $SC$.

The last equation describes the evolution of the market-level error term $%
\varepsilon _{m}$ over time. $\tilde{\varepsilon}_{m}$ is a pure white noise
that is independent across period. $\tau $ measures the persistence of the
unobserved market features. Notice that both $\tau $ and the sunk cost $SC$
can generate history dependence, but they have different implications.
Consider two markets $A$ and $B$ that have a similar market size today with
the same number of chain stores:$~X_{A}=X_{B},~D_{i,A}=D_{i,B},~i\in
\{k,w\}. $ Market $A$ used to be much bigger, but has gradually decreased in
size. The opposite is true for Market $B$: it was smaller before, but has
expanded over time. If history does not matter\ (that is, both $\tau $ are $%
SC$ are zero), these two markets should on average have the same number of
small stores. However, if $SC$ is important, then market $A$ should have
more small stores that entered in the past and have maintained their
business after the entry of chain stores. In other words, big sunk cost
implies that everything else equal, markets that were bigger historically
carry more small stores in the current period. On the other hand, if $\tau $
is important, then some markets have more small stores throughout the time,
but there are no systematic patterns between the market size in the previous
period and the number of stores in the current period, as the history
dependence is driven by the unobserved market shock $\varepsilon _{m}$ that
is assumed to be independent of $X_{m}.$

The market-level error term $\varepsilon _{m}$ makes the location choices of
the chain stores $D_{k,m}$ and $D_{w,m}$, and the number of small firms $%
N_{s,m}$ endogenous in the profit functions, since a large $\varepsilon _{m}$
leads to more entries of both chains and small firms. The chain effect $%
\delta _{ii}\frac{D_{i,l}}{Z_{ml}}$ is also endogenous, because a large $%
\varepsilon _{m}$ is associated with a high value of $D_{i,m},$ which
increases the profitability of market $l$, and hence leads to a high value
of $D_{i,l}.$ I solve the chains' and small firms' entry decisions
simultaneously within the model, and require the model to replicate the
location patterns observed in the data.

Note that the above specification allows very flexible competition patterns
among all the possible firm-pair combinations. The parameters to be
estimated are $\{\beta _{i},\delta _{ij},\delta _{ii},\rho ,\tau
,SC\},i,j\in \{k,w,s\}.$

\section{Solution algorithm}

The unobserved market level profit shock $\varepsilon _{m}$, together with
the chain effect $\delta _{ii}\Sigma _{l\neq m}\frac{D_{i,l}}{Z_{ml}}$,
renders all of the discrete variables $%
N_{s,m}^{0},~D_{i,m},~D_{j,m},~D_{i,l},~$and $N_{s,m}$ endogenous in the
profit functions (\ref{Profit_0}) and (\ref{Profit}). Finding the Nash
equilibrium of this game is complicated. I take several steps to address
this problem. Section \ref{OneAgent} explains how to solve each chain's
single agent problem, section \ref{Comp} derives the solution algorithm for
the game between two chains, and section \ref{SmallCom} adds the small
retailers and solves for the Nash equilibrium of the full model$.$

\subsection{Chain $i$'s single agent problem\label{OneAgent}}

In this subsection, let us focus on the chain's single-agent problem and
abstract from competition. In the next two subsections I incorporate
competition and solve the model for all players.

For notational simplicity, I have suppressed the firm subscript $i$ and used 
$X_{m}$ instead of $X_{m}\beta _{i}+\sqrt{1-\rho ^{2}}\varepsilon _{m}+\rho
\eta _{i,m}$ in the profit function throughout this subsection. Let $M$
denote the total number of markets, and let $\mathbf{D}=\{0,1\}^{M}$ denote
the choice set. An element of the set $\mathbf{D}$ is an $M$-coordinate
vector $D=\{D_{1},...,D_{M}\}$. The profit-maximization problem is:

\begin{equation*}
\max\limits_{D_{1,}...,D_{M}\in \{0,1\}}\Pi =\sum_{m=1}^{M}\left[ D_{m}\ast
(X_{m}+\delta \Sigma _{l\neq m}\frac{D_{l}}{Z_{ml}})\right]
\end{equation*}%
The choice variable $D_{m}$\ appears in the profit function in two ways.
First, it directly determines profit in market $m$: the firm earns $%
X_{m}+\delta \Sigma _{l\neq m}\frac{D_{l}}{Z_{ml}}$\ if $D_{m}=1,$\ and zero
if $D_{m}=0.$\ Second, the decision to open a store in market $m$\ increases
the profits in other markets through the chain effect.

The complexity of this maximization problem is twofold: first, it is a
discrete problem of a large dimension. In the current application, with $%
M=2065$ and two choices for each market (enter or stay outside), the number
of possible elements in the choice set $\mathbf{D}$ is $2^{2065},$ or
roughly $10^{600}.$ The naive approach that evaluates all of them to find
the profit-maximizing vector(s) is infeasible. Second, the profit function
is irregular:\ it is neither concave nor convex. Consider the function where 
$D_{m}$\ takes real values, rather than integers $\{0,$ $1\}$. The Hessian
of this function is indefinite, and the usual first-order condition does not
guarantee an optimum.\footnote{%
A symmetric matrix is positive (negative) semidefinite iff all the
eigenvalues are non-negative (non-positive). The Hessian of the profit
function (\ref{Profit}) is a symmetric matrix with zero for all the diagonal
elements. Its trace, which is equal to the sum of the eigenvalues, is zero.
If the Hessian matrix has a positive eigenvalue, it has to have a negative
one as well. There is only one possibility for the Hessian to be positive
(or negative) semidefinite, which is that all the eigenvalues are 0. This is
true only for the zero matrix H=0.} Even if one could exploit the
first-order condition, the search with a large number of choice variables is
a daunting task.

Instead of solving the problem directly, I\ transform it into a search for
the fixed points of the necessary conditions for profit maximization. In
particular, I exploit the lattice structure of the set of fixed points of an
increasing function and propose an algorithm that obtains an upper bound $%
D^{U}$ and a lower bound $D^{L}$ for the profit-maximizing vector(s)$\,$.
With these two bounds at hand, I evaluate all vectors that lie between them
to find the profit-maximizing location choice.

Throughout this paper, the comparison between vectors is coordinate-wise. A
vector $D$ is bigger than vector $D^{\prime }$ if and only if every element
of $D$ is weakly bigger: $D\geq D^{\prime }$ if and only if $D_{m}\geq
D_{m}^{\prime }$ $\forall m.$ $D$ and $D^{\prime }$ are unordered if neither 
$D\geq D^{\prime }$\ nor $D\leq D^{\prime }.$ They are the same if both $%
D\geq D^{\prime }$\ and $D\leq D^{\prime }.$

Let the profit maximizer be denoted $D^{\ast }=\arg \max_{D\in \mathbf{D}%
}\Pi (D)$. The optimality of $D^{\ast }$ implies that profit at $D^{\ast }$
must be (weakly) higher than the profit at any one-market deviation:%
\begin{equation*}
\Pi (D_{1}^{\ast },...,D_{m}^{\ast },...,D_{M}^{\ast })\geq \ \Pi
(D_{1}^{\ast },...,D_{m},...,D_{M}^{\ast }),\forall m
\end{equation*}%
which leads to:%
\begin{equation}
D_{m}^{\ast }=1[X_{m}+2\delta \Sigma _{l\neq m}\frac{D_{l}^{\ast }}{Z_{ml}}%
\geq 0],\forall m  \label{Nece}
\end{equation}%
The derivation of equation (\ref{Nece}) is left to appendix B.1. These
conditions have the usual interpretation that $X_{m}+2\delta \Sigma _{l\neq
m}\frac{D_{l}^{\ast }}{Z_{ml}}$ is market $m$'s marginal contribution to
total profit. This equation system is not definitional; it is a set of
necessary conditions for the optimal vector $D^{\ast }$. Not all vectors
that satisfy (\ref{Nece}) maximize profit, but if $D^{\ast }$ maximizes
profit, it must satisfy these constraints.

Define $V_{m}(D)=1[X_{m}+2\delta \Sigma _{l\neq m}\frac{D_{l}}{Z_{ml}}\geq
0],$ and $V(D)=\{V_{1}(D),...,V_{M}(D)\}.$ $V(\cdot )$ is a vector function
that maps from $\mathbf{D}$ into itself: $V:\mathbf{D}\rightarrow \mathbf{D}%
. $ It is an increasing function: $V(D^{\prime })\geq V(D^{\prime \prime })$
whenever $D^{\prime }\geq D^{\prime \prime }$, as $\delta _{ii}$ is assumed
non-negative$.$ By construction, the profit maximizer $D^{\ast }$ is one of $%
V(\cdot )$'s fixed points. The following theorem, proved by Tarski (1955),
states that the set of fixed points of an increasing function that maps from
a lattice into itself is a lattice and has a greatest point and a least
point. Appendix B.2 describes the basic lattice theory.

\begin{theorem}
\label{Tarski} \textit{Suppose that }$Y(X)$\textit{\ is an increasing
function from a nonempty complete lattice }$\mathbf{X}$\textit{\ into\ }$%
\mathbf{X}.$

(a) \textit{The set of fixed points of }$Y(X)$\textit{\ is nonempty, }$\sup_{%
\mathbf{X}}(\{X\in \mathbf{X},X\leq Y(X)\})$\textit{\ is the greatest fixed
point, and }$\inf_{\mathbf{X}}(\{X\in \mathbf{X},Y(X)\leq X\})$\textit{\ is
the least fixed point.}

(b) \textit{The set of fixed points of }$Y(X)$\textit{\ in }$\mathbf{X}$%
\textit{\ is a nonempty complete lattice.}
\end{theorem}

A lattice in which each nonempty subset has a supremum and an infimum is
complete. Any finite lattice is complete. A nonempty complete lattice has a
greatest and a least element. Since the choice set $\mathbf{D}$ is a finite
lattice, it is complete, and Theorem \ref{Tarski} can be directly applied.
Several points are worth mentioning. First, $\mathbf{X}$ can be a closed
interval or it can be a discrete set, as long as the set includes the
greatest lower bound and the least upper bound for any of its nonempty
subsets. That is, it is a complete lattice. Second, the set of fixed points
is itself a nonempty complete lattice, with a greatest and a smallest point.
Third, the requirement that $Y(X)$ is \textquotedblleft
increasing\textquotedblright\ is crucial; it can't be replaced by assuming
that $Y(X)$ is a monotone function. Appendix B.2 provides a counterexample
where the set of fixed points for a decreasing function is empty.

Now I outline the algorithm that delivers the greatest and the least fixed
point of $V(D)$, which are, respectively, an upper bound and a lower bound
for the optimal solution vector $D^{\ast }.$ To find $D^{\ast }$, I rely on
an exhaustive search among the vectors lying between these two bounds.

Start with $D^{0}=\sup (\mathbf{D})=\{1,...,1\}.$ The supremum exists
because $\mathbf{D}$ is a complete lattice. Define a sequence $\{D^{t}\}:$ $%
D^{1}=V(D^{0}),$ and $D^{t+1}=V(D^{t}).$ By construction, we have: $%
D^{0}\geq V(D^{0})=D^{1}.$ Since $V(\cdot )$ is an increasing function, $%
V(D^{0})\geq V(D^{1}),$ or $D^{1}\geq D^{2}.$ Iterating this process several
times generates a decreasing sequence: $D^{0}\geq D^{1}\geq ...\geq D^{t}.$
Given that $D^{0}$ has only $M$ distinct elements and at least one element
of the $D$ vector is changed from 1 to 0 in each iteration, the process
converges within $M$ steps: $D^{T}=D^{T+1},T\leq M.$ Let $D^{U}$ denote the
convergent vector. $D^{U}$ is a fixed point of the function $V(\cdot
):D^{U}=V(D^{U}).$ To show that $D^{U}$ is indeed the greatest element of
the set of fixed points, note that $D^{0}\geq D^{\prime },$ where $D^{\prime
}$ is an arbitrary element of the set of fixed points. Applying the function 
$V(\cdot )$ to the inequality $T$ times, we have $D^{U}=V^{T}(D^{0})\geq
V^{T}(D^{\prime })=D^{\prime }.$

Using the dual argument, one can show that the convergent vector derived
from $D^{0}=\inf (\mathbf{D})=\{0,...,0\}$ is the least element in the set
of fixed points. Denote it by $D^{L}.$ In appendix B.3, I show that starting
from the solution to a constrained version of the profit maximization
problem yields a tighter lower bound. There I also illustrate how a tighter
upper bound can be obtained by starting with a vector $\tilde{D}$ such that $%
\tilde{D}\geq D^{\ast }$ and $\tilde{D}\geq V(\tilde{D}).$

With the two bounds $D^{U}$ and $D^{L}$ at hand, I evaluate all vectors that
lie between them and find the profit-maximizing vector $D^{\ast }.$

\subsection{The maximization problem with two competing chains\label{Comp}}

The discussion in the previous subsection abstracts from rival-chain
competition and considers only the chain effect. With the competition from
the rival chain, the profit function for chain $i$ becomes: $\Pi
_{i}(D_{i},D_{j})=\Sigma _{m=1}^{M}[D_{i,m}\ast (X_{im}+\delta _{ii}\Sigma
_{l\neq m}\frac{D_{i,l}}{Z_{ml}}+\delta _{ij}D_{j,m})],$ where $X_{im}$
contains $X_{m}\beta _{i}+\sqrt{1-\rho ^{2}}\varepsilon _{m}+\rho \eta
_{i,m}.$

To address the interaction between the chain effect and the competition
effect, I invoke the following theorem from Topkis\ (1978), which states
that the best response function is decreasing in the rival's strategy when
the payoff function is supermodular and has decreasing differences.
Specifically:\footnote{%
The original theorem is stated in terms of $\Pi (D,t)$ having increasing
differences in $(D,t),$ and $\arg \max_{D\in \mathbf{D}}\Pi (D,t)$
increasing in $t.$ Replacing $t$ with $-t$ yields the version of the theorem
stated here.}$^{,}$\footnote{%
See Milgrom and Shannon (1994) for a detailed discussion on the necessary
and sufficient condition for the solution set of an optimization problem to
be monotonic in the parameters.}

\begin{theorem}
\label{Topkis} \textit{If $\mathbf{X}$\ is a lattice, }$K$\textit{\ is a
partially ordered set, }$Y(X,k)$\textit{\ is supermodular in }$X$\textit{\
on }$\mathbf{X}$\textit{\ for each }$k$\textit{\ in }$K$\textit{, and }$%
Y(X,k)$\textit{\ has decreasing differences in }$(X,k)$\textit{\ on }$%
\mathbf{X}\times K$\textit{, then }$\arg \max_{X\in \mathbf{X}}Y(X,k)$%
\textit{\ is decreasing in }$k$\textit{\ on }$\{k:k\in K,\arg \max_{X\in 
\mathbf{X}}Y(X,k)$\textit{\ is nonempty}$\}.$
\end{theorem}

$Y(X,k)$ has decreasing differences in $(X,k)$ on $\mathbf{X}\times K$ if $%
Y(X,k^{\prime \prime })-Y(X,k^{\prime })$ is decreasing in $X\in \mathbf{X}$
for all $k^{\prime }\leq k^{\prime \prime }$ in $K.$ Intuitively, $Y(X,k)$
has decreasing differences in $(X,k)$ if $X$ and $k$ are substitutes. In
appendix B.4, I verify that the profit function $\Pi
_{i}(D_{i},D_{j})=\Sigma _{m=1}^{M}[D_{i,m}\ast (X_{im}+\delta _{ii}\Sigma
_{l\neq m}\frac{D_{i,l}}{Z_{ml}}+\delta _{ij}D_{j,m})]$ is supermodular in
its own strategy $D_{i}$ and has decreasing differences in $(D_{i},D_{j}).$
From Theorem \ref{Topkis}, chain $i$'s best response function$~\arg
\max_{D_{i}\in \mathbf{D}_{\mathbf{i}}}\Pi _{i}(D_{i},D_{j})$ decreases in
rival $j$'s strategy $D_{j}$.\ Similarly for chain $j$'s best response to $i$%
's strategy.

The set of Nash equilibria of a supermodular game is nonempty and it has a
greatest element and a least element.\footnote{%
See Topkis (1978) and Zhou (1994).}$^{,}$\footnote{%
A game is supermodular if the payoff function $\Pi _{i}(D_{i},D_{-i})$ is
supermodular in $D_{i}$ for each $D_{-i}$ and each player $i$, and $\Pi
_{i}(D_{i},D_{-i})$ has increasing differences in $(D_{i},D_{-i})$ for each $%
i.$}\ The current entry game is not supermodular, as the profit function has
decreasing differences in the joint strategy space $\mathbf{D}\times \mathbf{%
D}$. This leads to a non-increasing joint best response function, and we
know from the discussion after Theorem \ref{Tarski} that a non-increasing
function on a lattice can have an empty set of fixed points. A simple
transformation, however, restores the supermodularity property of the game.
The trick is to define a new strategy space for one player (for example,
Kmart) to be the negative of the original space. Let $\widetilde{\mathbf{D}}%
_{k}=-\mathbf{D}_{k}$. The profit function can be re-written as: 
\begin{eqnarray*}
\Pi _{k}(-D_{k},D_{w}) &=&\Sigma _{m}(-D_{k,m})\ast \lbrack -X_{km}+\delta
_{kk}\Sigma _{l\neq m}\frac{-D_{k,l}}{Z_{ml}}+(-\delta _{kw})D_{w,m}] \\
\Pi _{w}(D_{w},-D_{k}) &=&\Sigma _{m}D_{w,m}\ast \lbrack X_{wm}+\delta
_{ww}\Sigma _{l\neq m}\frac{D_{w,l}}{Z_{ml}}+(-\delta _{wk})(-D_{k,m})]
\end{eqnarray*}%
It is easy to verify that the game defined on the new strategy space $(%
\widetilde{\mathbf{D}}_{k},\mathbf{D}_{w})$ is supermodular, therefore a
Nash equilibrium exists. Using the transformation $\widetilde{\mathbf{D}}%
_{k}=-\mathbf{D}_{k}$, one can find the corresponding equilibrium in the
original strategy space. In the following paragraphs, I explain how to find
the desired Nash equilibrium directly in the space of $(\mathbf{D}_{k},%
\mathbf{D}_{w})$ using the \textquotedblleft Round-Robin\textquotedblright\
algorithm, where each player proceeds in turn to update its own strategy.%
\footnote{%
See page 185 of Topkis (1998) for a detailed discussion.}

To obtain the equilibrium most profitable for Kmart, start with the smallest
vector in Wal-Mart's strategy space: $D_{w}^{0}=\inf (\mathbf{D}%
)=\{0,...,0\}.$ Derive Kmart's best response $K(D_{w}^{0})=\arg
\max_{D_{k}\in \mathbf{D}}\Pi _{k}(D_{k},D_{w}^{0})$ given $D_{w}^{0}$,
using the method outlined in section \ref{OneAgent}, and denote it by $%
D_{k}^{1}=K(D_{w}^{0}).$ Similarly, find Wal-Mart's best response $%
W(D_{k}^{1})=\arg \max_{D_{w}\in \mathbf{D}}\Pi _{w}(D_{w},D_{k}^{1})$ given 
$D_{k}^{1}$, again using the method in section \ref{OneAgent}, and denote it
by $D_{w}^{1}.$ Note that $D_{w}^{1}\geq D_{w}^{0},$ by the construction of $%
D_{w}^{0}\mathbf{.}$ This finishes the first iteration $%
\{D_{k}^{1},D_{w}^{1}\}.$

Fix $D_{w}^{1}$ and solve for Kmart's best response $D_{k}^{2}=K(D_{w}^{1}).$
By Theorem \ref{Topkis}, Kmart's best response decreases in the rival's
strategy, so $D_{k}^{2}=K(D_{w}^{1})\leq D_{k}^{1}=K(D_{w}^{0}).$ The same
argument shows that $D_{w}^{2}\geq D_{w}^{1}.$ Iterating this process
generates two monotone sequences: $D_{k}^{1}\geq D_{k}^{2}\geq ...\geq
D_{k}^{t},~D_{w}^{1}\leq D_{w}^{2}\leq ...\leq D_{w}^{t}.$ In every
iteration, at least one element of the $D_{k}$ vector is changed from 1 to
0, and one element of the $D_{w}$ vector is changed from 0 to 1, so the
algorithm converges within $M$ steps: $%
D_{k}^{T}=D_{k}^{T+1},D_{w}^{T}=D_{w}^{T+1},T\leq M.$ The convergent vectors 
$(D_{k}^{T},D_{w}^{T})$ constitute an equilibrium: $%
D_{k}^{T}=K(D_{w}^{T}),D_{w}^{T}=W(D_{k}^{T}).$ Furthermore, this
equilibrium gives Kmart the highest profit among the set of all equilibria.

That Kmart prefers the equilibrium $(D_{k}^{T},D_{w}^{T})$ obtained using $%
D_{w}^{0}=\{0,...,0\}$ to all other equilibria follows from two results:
first, $D_{w}^{T}\leq D_{w}^{\mathbf{\ast }}$ for any $D_{w}^{\mathbf{\ast }%
} $ that belongs to an equilibrium; second, $\Pi _{k}(K(D_{w}),D_{w})$
decreases in $D_{w}$, where $K(D_{w})$ denotes Kmart's best response
function. Together they imply that $\Pi _{k}(D_{k}^{T},D_{w}^{T})\geq \Pi
_{k}(D_{k}^{\ast },D_{w}^{\ast }),~\forall ~\{D_{k}^{\ast },D_{w}^{\ast }\}$
that belongs to the set of Nash equilibria.

To show the first result, note that $D_{w}^{0}\leq D_{w}^{\ast }$, by the
construction of $D_{w}^{0}$. Since $K(D_{w})$ decreases in $D_{w},$ $%
D_{k}^{1}=K(D_{w}^{0})\geq K(D_{w}^{\ast })=D_{k}^{\ast }.$ Similarly, $%
D_{w}^{1}=W(D_{k}^{1})\leq W(D_{k}^{\ast })=D_{w}^{\ast }.$ Repeating this
process $T$ times leads to $D_{k}^{T}=K(D_{w}^{T})\geq $ $K(D_{w}^{\ast
})=D_{k}^{\ast },$ and $D_{w}^{T}=W(D_{k}^{T})\leq W(D_{k}^{\ast
})=D_{w}^{\ast }.$ The second result follows from $\Pi _{k}(K(D_{w}^{\ast
}),D_{w}^{\ast })\leq \Pi _{k}(K(D_{w}^{\ast }),D_{w}^{T})\leq \Pi
_{k}(K(D_{w}^{T}),D_{w}^{T}).$ The first inequality holds because Kmart's
profit function decreases in its rival's strategy, while the second
inequality follows from the definition of the best response function $%
K(D_{w}).$

By the dual argument, starting with $D_{k}^{0}=\inf (\mathbf{D}%
)=\{0,...,0\}\ $delivers the equilibrium that is most preferred by Wal-Mart.
To search for the equilibrium that favors Wal-Mart in the southern region
and Kmart in the rest of the country, one uses the same algorithm to solve
the game separately for the south and the other regions.

\subsection{Adding small firms\label{SmallCom}}

It is straightforward to solve the \textquotedblleft
pre-chain\textquotedblright\ stage: $N_{s}^{0}$ is the largest integer such
that all entering firms can recover their sunk cost:\footnote{%
The number of potential small entrants is assumed to be 11, which was within
the top two-pencentile of the distribution of the number of small stores. I
also experimented with the maximum number of small stores throughout the
sample period. See footnote $^{\text{\ref{SmNoFtNote}}}$ for details.}%
\begin{equation*}
\Pi _{s,m}^{0}=X_{m}^{0}\beta _{s}+\delta _{ss}\ln (N_{s,m}^{0})+\sqrt{%
1-\rho ^{2}}\varepsilon _{m}^{0}+\rho \eta _{s,m}^{0}-SC>0
\end{equation*}

After the entry of chain stores, some of the existing small stores will find
it unprofitable to compete with chains and exit the market, while other more
efficient stores (the ones with larger $\eta _{s,m}$) will enter the market
after paying the sunk cost of entry. The number of small stores in the
\textquotedblleft post-chain\textquotedblright\ period is a sum of new
entrants $N_{s}^{E}$ and the remaining incumbents $N_{s}^{I}$. Except for
the idiosyncratic profit shocks, the only difference between these two
groups of small firms is the sunk cost: 
\begin{equation*}
\Pi _{s,m}=X_{m}\beta _{s}+\sum_{i=k,w}\delta _{si}D_{i,m}+\delta _{ss}\ln
(N_{s,m})+\sqrt{1-\rho ^{2}}\varepsilon _{m}+\rho \eta _{s,m}-SC\ast 1[\text{%
new entrant}]
\end{equation*}%
Potential entrants will enter the market only if the \textquotedblleft
post-chain\textquotedblright\ profit can recover the sunk cost, while
existing small firms will maintain the business as long as the profit is
non-negative.

Both the number of entrants $N_{s}^{E}(D_{k},D_{w})$ and the number of
remaining incumbents $N_{s}^{I}(D_{k},D_{w})$ are well defined functions of
the number of chain stores. To solve the game between chains and small
stores in the \textquotedblleft post-chain\textquotedblright\ period, I
follow the standard backward induction, and plug in small stores' reaction
functions to the chains' profit function. Specifically, chain $i$'s profit
function now becomes $\Pi _{i}(D_{i},D_{j})=\Sigma _{m=1}^{M}[D_{i,m}\ast
(X_{im}+\delta _{ii}\Sigma _{l\neq m}\frac{D_{i,l}}{Z_{ml}}+\delta
_{ij}D_{j,m}+\delta _{is}\ln
(N_{s}^{E}(D_{i,m},D_{j,m})+N_{s}^{I}(D_{i,m},D_{j,m})+1)]$, where $X_{im}$
is defined in subsection \ref{Comp}. The profit function $\Pi
_{i}(D_{i},D_{j})$ remains supermodular in $D_{i}$ with decreasing
differences in $(D_{i},D_{j})$ under a minor assumption, which essentially
requires that the net competition effect of rival $D_{j}$ on chain $i$'s
profit is negative.\footnote{%
If we ignore the integer problem and the sunk cost, then $\delta _{ss}\ln
(N_{s}+1)$ can be approximated by $-(X_{sm}+\delta _{sk}D_{k}+\delta
_{sw}D_{w})$, and the assumption is: $\delta _{kw}-\frac{\delta _{ks}\delta
_{sw}}{\delta _{ss}}<0,\delta _{wk}-\frac{\delta _{ws}\delta _{sk}}{\delta ss%
}<0.$ The expression is slightly more complicated with the integer
constraint and the distinction between existing small stores and new
entrants. Essentially, these two conditions imply that when there are small
stores, the `net' competition effect of Wal-Mart (its direct impact,
together with its indirect impact working through small stores) on Kmart's
profit and that of Kmart on Wal-Mart's profit are still negative. I have
verified in the empirical application that these conditions are indeed
satisfied.}

The main computational burden in solving the full model with both chains and
small retailers is the search for the best responses $K(D_{w})$ and $%
W(D_{k}).$ In appendix B.5, I\ discuss a few technical details related with
the implementation.

\section{Empirical implementation\label{Empirical}}

\subsection{Estimation}

The model does not yield a closed form solution to firms' location choices
conditioning on market size observables and a given vector of parameter
values. Hence I turn to simulation methods. The ones most frequently used in
the I.O. literature are the method of simulated log-likelihood (MSL) and the
method of simulated moments (MSM).

Implementing MSL is difficult because of the complexities in obtaining an
estimate of the log-likelihood of the observed sample. The cross-sectional
dependence among the observed outcomes in different markets indicates that
the log-likelihood of the sample is no longer the sum of the log-likelihood
of each market, and one needs an exceptionally large number of simulations
to get a reasonable estimate of the sample's likelihood. Thus I adopt the
MSM method to estimate the parameters in the profit functions $\theta
_{0}=\{\beta _{i},\delta _{ii},\delta _{ij},\rho ,\tau ,SC\}_{i=k,w,s}\in 
\mathbf{\Theta \subset R}^{P}.$ The following moment condition is assumed to
hold at the true parameter value $\theta _{0}$:%
\begin{equation*}
E[g(X_{m},\theta _{0})]=0
\end{equation*}%
where $g(X_{m},\cdot )\in \mathbf{R}^{L}~$with $L\geq P$ is a vector of
moment functions that specifies the differences between the observed
equilibrium market structures and those predicted by the model.

A MSM estimator $\hat{\theta}$ minimizes a weighted quadratic form in $%
\Sigma _{m=1}^{M}\hat{g}(X_{m},\theta ):$%
\begin{equation}
\theta =\arg \min_{\theta \in \Theta }\frac{1}{M}\left[ \tsum%
\limits_{m=1}^{M}\hat{g}(X_{m},\theta )\right] ^{\prime }\mathbf{\Omega }%
_{M}~\left[ \tsum\limits_{m=1}^{M}\hat{g}(X_{m},\theta )\right]  \label{Obj}
\end{equation}%
where $\hat{g}(\cdot )$ is a simulated estimate of the true moment function,
and $\mathbf{\Omega }_{M}$ is an $L\times L$ positive semi-definite
weighting matrix. Assume $\mathbf{\Omega }_{M}\overset{p}{\rightarrow }%
\mathbf{\Omega }_{0}$, an $L\times L$ positive definite matrix. Define the $%
L\times P$ matrix $\mathbf{G}_{0}=E[\nabla _{\theta ^{\prime
}}g(X_{m},\theta _{0})].$ Under some mild regularity conditions, Pakes and
Pollard (1989) and McFadden (1989) show that:%
\begin{equation}
\sqrt{M}(\hat{\theta}-\theta _{0})\overset{d}{\rightarrow }\text{Normal}(%
\mathbf{0},(1+R^{-1})\ast \mathbf{A}_{0}^{-1}\mathbf{B}_{0}\mathbf{A}%
_{0}^{-1})  \label{Distri}
\end{equation}%
where $R$ is the number of simulations, $\mathbf{A}_{0}\equiv \mathbf{G}%
_{0}^{\prime }\mathbf{\Omega }_{0}\mathbf{G}_{0},\ \mathbf{B}_{0}=\mathbf{G}%
_{0}^{\prime }\mathbf{\Omega }_{0}\mathbf{\Lambda }_{0}\mathbf{\Omega }_{0}%
\mathbf{G}_{0},$ and $\mathbf{\Lambda }_{0}=E[g(X_{m},\theta
_{0})g(X_{m},\theta _{0})^{\prime }]=$Var$[g(X_{m},\theta _{0})].$ If a
consistent estimator of $\mathbf{\Lambda }_{0}^{-1}$ is used as the
weighting matrix, the MSM estimator $\hat{\theta}$ is asymptotically
efficient,\footnote{%
The MSM estimator $\hat{\theta}$ is asymptotically efficient relative to
estimators which minimize a quadratic norm in $g(\cdot ).$ Different moments
could improve efficiency. The author thanks the referee for pointing this
out.} with its asymptotic variance being Avar$(\hat{\theta})=(1+R^{-1})\ast (%
\mathbf{G}_{0}^{\prime }\mathbf{\Lambda }_{0}^{-1}\mathbf{G}_{0})^{-1}/M.$

The obstacle in using this standard method is that the moment functions $%
g(X_{m},\cdot )$ are no longer independent across markets when the chain
effect induces spatial correlation in the equilibrium outcome. For example,
Wal-Mart's entry decision in Benton County, Arkansas directly relates to its
entry decision in Carroll County, Arkansas, Benton's neighbor. In fact, any
two entry decisions, $D_{i,m}$ and $D_{i,l}$, are correlated because of the
chain effect, although the dependence becomes very weak when market $m$ and
market $l$ are far apart, since the benefit $\frac{D_{i,l}}{Z_{ml}}$
evaporates with distance. As a result, the covariance matrix in equation (%
\ref{Distri}) is no longer valid.

Conley (1999) discussed method of moments estimators using data that exhibit
spatial dependence. The paper provided sufficient conditions for consistency
and normality, which require the underlying data generating process to
satisfy a strong mixing condition.\footnote{%
The asymptotic arguments require the data to be generated from locations
that grow uniformly in spatial dimensions as the sample size increases.}
Essentially, the dependence among observations should die away quickly as
the distance increases. In the current application, the consistency
condition requires that the covariance between $D_{i,m}$ and $D_{i,l}$ goes
to $0$ as their geographic distance increases.\footnote{%
Here I\ briefly verify that the consistency condition is satisfied. By
construction:%
\begin{eqnarray*}
D_{i,m} &=&1[X_{i,m}+2\delta _{ii}\frac{D_{i,l}}{Z_{ml}}+\rho \eta
_{i,m}\geq 0] \\
D_{i,l} &=&1[X_{i,l}+2\delta _{ii}\frac{D_{i,m}}{Z_{ml}}+\rho \eta
_{i,l}\geq 0]
\end{eqnarray*}%
where $X_{i,m}=X_{m}\beta _{i}+2\delta _{ii}\sum_{k\neq l,m}\frac{D_{i,k}}{%
Z_{mk}}+\delta _{ij}D_{j,m}+\delta _{is}\ln (N_{s,m}+1)+\sqrt{1-\rho ^{2}}%
\varepsilon _{m}.$ The covariance between $D_{i,m}$ and $D_{i,l}$ is:%
\begin{eqnarray*}
cov(D_{i,m},D_{i,l}) &=&E(D_{i,m}\ast D_{i,l})-E(D_{i,m})\ast E(D_{i,l}) \\
&\leq &\Pr (\rho \eta _{i,m}\geq -X_{i,m}-\frac{2\delta _{ii}}{Z_{ml}},~\rho
\eta _{i,l}\geq -X_{i,l}-\frac{2\delta _{ii}}{Z_{ml}})- \\
&&\Pr (\rho \eta _{i,m}\geq -X_{i,m})\ast \Pr (\rho \eta _{i,l}\geq -X_{i,l})
\\
&\rightarrow &0\text{ as }Z_{ml}\rightarrow \infty
\end{eqnarray*}%
} In other words, the entry decisions in different markets should be nearly
independent when the distance between these markets are sufficiently large.

Unlike some other iteration procedures that search for the fixed points, $%
\left\vert \delta _{ii}\right\vert $ does not have to be less than 1. To see
this, note that by construction, 
\begin{equation*}
D_{i,m}=1[X_{i,m}+2\delta _{ii}\Sigma _{l\neq m}\frac{D_{i,l}}{Z_{ml}}\geq
0],\forall m
\end{equation*}%
where $D_{i,m}=1$ if chain $i$ has a store in market $m$. The system stays
stable as long as $\delta _{ii}$ is finite, because $D_{i,m}$ is bounded $%
\forall m$. The geographic scope of the spillover effect can increase with
the sample size, but the sum $\delta _{ii}\sum_{l\neq m}\frac{D_{i,l}}{Z_{lm}%
}$ should remain finite, to prevent the profit function from exploding.
There are many ways to formulate the relationship between the spillover
effect and the distance, as long as it guarantees that the (pair-wise)
covariance between the chain stores' entry decisions in different markets
goes to $0$ as the geographic distance increases.\footnote{%
The normality conditions in Conley (1999) require the covariance to decrease
at a sufficiently fast rate. See page 9 of the paper for details. These
conditions are trivially satisfied here, as the spillover effect is assumed
to occur only within 50 or 100 miles. In other applications, one needs to
verify that these conditions are satisfied.}

With the presence of the spatial dependence, the asymptotic covariance
matrix of the moment functions $\mathbf{\Lambda }_{0}$ in equation (\ref%
{Distri}) should be replaced by $\mathbf{\Lambda }_{0}^{d}=\Sigma _{s\in
M}E[g(X_{m},\theta _{0})g(X_{s},\theta _{0})^{\prime }].$ Conley (1999)
proposes a nonparametric covariance matrix estimator formed by taking a
weighted average of spatial autocovariance terms, with zero weights for
observations farther than a certain distance. Following Conley (1999) and
Conley \& Ligon (2002), the estimator of $\Lambda _{0}^{d}$ is:%
\begin{equation}
\hat{\Lambda}\equiv \frac{1}{M}\Sigma _{m}\Sigma _{s\in B_{m}}\left[ \hat{g}%
(X_{m},\theta )\hat{g}(X_{s},\theta )^{\prime }\right]  \label{Var}
\end{equation}%
where $B_{m}$ is the set of markets whose centroid is within fifty miles of
market $m$, including market $m.\footnote{%
As mentioned in Conley (1999), this estimator is inefficient and not always
positive semi-definite. Newey and West (1987) introduced a weight function $%
w(l,m)$ as a numerical device to make the estimator positive semi-definite.
The weight used in the empirical application is 0.5 for all the neighbors.}$
The implicit assumption is that the spillover effect is negligible for
markets beyond fifty miles. I have also estimated the variance of the moment
functions $\hat{\Lambda}$\ summing over markets within a hundred miles. All
of the parameters that are significant with the smaller set of $B_{m}$
remain significant, and the changes in the t-statistics are small.

The estimation procedure is as follows.

Step one: start from some initial guess of the parameter values, and draw
independently from the normal distribution the following vectors:\ the
market-level errors for both the \textquotedblleft
pre-chain\textquotedblright\ period and the \textquotedblleft
post-chain\textquotedblright\ period: $\{\varepsilon _{m}^{0}\}_{m=1}^{M}$
and $\{\tilde{\varepsilon}_{m}\}_{m=1}^{M};$ profit shocks for the chains: $%
\{\eta _{k,m}\}_{m=1}^{M},\{\eta _{w,m}\}_{m=1}^{M};$ and profit shocks for
each of the potential small entrant: $\{\eta _{s,m}^{0}\}_{m=1}^{M},$ and $%
\{\eta _{s,m}\}_{m=1}^{M},$ where $s=1,...,11.$\footnote{%
The number of potential small entrants is assumed to be 11. During the
sample period, only one county had 25 small stores, while the median number
was 4 for the 1970s, and 3 for the 1980s and 1990s. As the memory
requirement and the computational burden increase with the number of
potential entrant, I\ have capped the maximum number of small stores at 11,
which is within the top one-percentile of the distribution in the 1990s and
the top two-percentile in the 1980s. The competition effects of chains on
small stores do not change much with a maximum number of 25 small stores.%
\label{SmNoFtNote}}

Step two: obtain the simulated profits $\hat{\Pi}_{i},i=k,w,s$ and solve for 
$\hat{N}_{s}^{0},\hat{D}_{k},\hat{D}_{w},$ and $\hat{N}_{s}$.

Step three:\ repeat step one and two $R$ times and formulate $\hat{g}%
(X_{m},\theta ).$ Search for parameter values that minimize the objective
function (\ref{Obj}), while using the same set of simulation draws for all
values of $\theta $. To implement the two-step efficient estimator, I
substitute a preliminary estimate $\tilde{\theta}$ into equation (\ref{Var})
to compute the optimal weight matrix $\hat{\Lambda}^{-1}$ for the second
step.

Instead of the usual machine-generated pseudo-random draws, I\ use Halton
draws, which have better coverage properties and smaller simulation
variances.\footnote{%
A Halton sequence is defined in terms of a given number, usually a prime. As
an illustration, consider the prime 3. Divide the unit interval evenly into
three segments. The first two terms in the Halton sequence are the two break
points: $\frac{1}{3}$ and $\frac{2}{3}.$ Then divide each of these three
segments into thirds, and add the break points for these segments into the
sequence in a particular way: $\frac{1}{3},\frac{2}{3},\frac{1}{9},\frac{4}{9%
},\frac{7}{9},\frac{2}{9},\frac{5}{9},\frac{8}{9}.$ Note that the lower
break points in all three segments ($\frac{1}{9},\frac{4}{9},\frac{7}{9})$
are entered in the sequence before the higher break points ($\frac{2}{9},%
\frac{5}{9},\frac{8}{9}).$ Then each of the 9 segments is divided into
thirds, and the break points are added to the sequence: $\frac{1}{3},\frac{2%
}{3},\frac{1}{9},\frac{4}{9},\frac{7}{9},\frac{2}{9},\frac{5}{9},\frac{8}{9},%
\frac{1}{27},\frac{10}{27},\frac{19}{27},\frac{4}{27},\frac{13}{27},\frac{22%
}{27},$ and so on. This process is continued for as many points as the
researcher wants to obtain. See chapter 9 of \textquotedblleft Discrete
Choice Methods with Simulation (2003)\textquotedblright\ by Kenneth Train
for an excellent discussion of the Halton draws.}$^{,}$\footnote{%
In situations of high-dimensional simulations (as is the case here), the
standard Halton draws display high correlations. The estimation here uses
shuffled halton draws, as proposed in Hess and Polak (2003), which has
documented that the high correlation can be easily removed by shuffling the
halton draws.}According to Train (2000), 100 Halton draws achieves greater
accuracy in his mixed logit estimation than 1000 pseudo-random draws. The
parameter estimation exploits 150 Halton simulation draws while the variance
is calculated with 300 Halton draws.

There are twenty-nine parameters with the following set of moments that
match the model-predicted and the observed values of a) numbers of Kmart
stores,\ Wal-Mart stores, and numbers of small stores in the
\textquotedblleft pre-chain\textquotedblright\ period as well as the
\textquotedblleft post-chain\textquotedblright\ period; b) various kinds of
market structures (for example, only a Wal-Mart store but no Kmart stores);
c) the number of chain stores in the nearby markets; d) the interaction
between the market size variables and the above items; e) the difference in
the number of small stores between the \textquotedblleft
pre-chain\textquotedblright\ and \textquotedblleft
post-chain\textquotedblright\ period, interacted with the changes in the
market size variables between these two periods.

\subsection{Discussion: a closer look at the assumptions and possible
extensions\label{Chain}}

Now I discuss several assumptions of the model: the game's information
structure and issues of multiple equilibria, the symmetry assumption for
small firms, and the non-negativity of the chain effect. In the last
subsection, I consider possible extensions.

\subsubsection{Information structure and multiple equilibria}

In the empirical entry literature, a common approach is to assume complete
information and simultaneous entry. One problem with this approach is the
presence of multiple equilibria, which has posed considerable challenges to
estimation. Some researchers look for features that are common among
different equilibria. For example, Bresnahan and Reiss (1990 and 1991) and
Berry (1992) point out that although firm identities differ across different
equilibria, the number of entering firms might be unique. Grouping different
equilibria by their common features leads to a loss of information and less
efficient estimates. Further, common features are increasingly difficult to
find when the model becomes more realistic.\footnote{%
For example, the number of entering firms in a given market is no longer
unique in the current application with the chain effect. See footnote $^{%
\text{\ref{NonUnique}}}$ for an illustration.} Others give up point
identification of parameters and search for bounds. These papers typically
involve bootstraps or subsampling, and are too computationally intensive to
be applicable here.\footnote{%
For example, the methods proposed in Andrews, Berry and Jia (2004),
Chernozhukov, Hong, and Tamer (2004), and Romano and Shaikh (2006), all
involve estimating the parameters for each bootstrap sample or subsample. It
takes more than a day to estimate the model once; it will take about a year
if 300 bootstrap samples or subsamples are used for inference. The method
proposed by Pakes, Porter, Ho, and Ishii (2005) is less computationally
demanding, but as the authors pointed out, the precision of their inference
is still an open question.}

Given the above considerations, I choose an equilibrium that seems
reasonable a priori. In the baseline specification, I estimate the model
using the equilibrium that is most profitable for Kmart because Kmart
derives from an older entity and historically might have had a first-mover
advantage. As a robustness check, I experiment with two other cases. The
first one chooses the equilibrium that is most profitable for Wal-Mart. This
is the direct opposite of the baseline specification and is inspired by the
hindsight of Wal-Mart's success. The second one selects the equilibrium that
is most profitable for Wal-Mart in the south and most profitable for Kmart
in the rest of the country. This is based on the observation that the
northern regions had been Kmart's backyard until recently while Wal-Mart
started its business from the south and has expertise in serving the
southern population. The estimated parameters for the different cases are
very similar to one another, which provides evidence that the results are
robust to the equilibrium choice. In section \ref{Parameter}, I also
investigate the differences between these equilibria at the estimated
parameter values.\footnote{%
If we were to formally test whether the data prefer one equilibrium to the
other, we need to derive the asymptotic distribution of the difference
between the two minimized objective function values, each associated with a
different equilibrium. It is a non-nested test that is somewhat involved as
one objective function contains moments with a nonzero mean at the true
parameter values. In the current application, the objective function values
are very similar in 1997 (108.68 for the objective function that uses the
equilibrium most profitable for Kmart, 105.02 for the equilibrium most
profitable for Wal-Mart, and 103.9 for the equilibrium that grants a
regional advantage to each player), but differ somewhat in 1988 (the
objective function values are 120.26, 120.77, and 136.74, respectively).
\par
{}} On average, they differ only in a small portion of the sample, and
results from the counter-factual exercises do not vary much across different
equilibria.

\subsubsection{The symmetry assumption for small firms}

I have assumed that all small firms have the same profit function and only
differ in the unobserved profit shocks. The assumption is necessitated by
data availability, since I\ do not observe any firm characteristics for
small firms. Making this assumption greatly simplifies the complexity of the
model with asymmetric competition effects, as it guarantees that in the
first and the third stage, the equilibrium number of small firms in each
market is unique.

\subsubsection{The chain effect $\protect\delta _{ii}$ \label{Delta}}

The assumption that $\delta _{ii}\geq 0,i\in \left\{ k,w\right\} $ is
crucial to the solution algorithm, since it implies that the function $V(D)$
defined by the necessary condition (\ref{Nece}) is increasing, and that the
profit function (\ref{Profit}) is supermodular in chain $i$'s own strategy.
These results allow me to employ two powerful theorems -- Tarski's fixed
point theorem and Topkis's monotonicity theorem -- to solve a complicated
problem that is otherwise unmanageable. The parameter $\delta _{ii}$ does
not have to be a constant. It can be region specific, or it can vary with
the size of each market (for example, interacting with population), as long
as it is weakly positive. However, the algorithm breaks down if either $%
\delta _{kk}$ or $\delta _{ww}$ becomes negative, and it excludes scenarios
where the chain effect is positive in some regions and negative in others.

The discussion so far has focused on the beneficial aspect of locating
stores close to each other. In practice, stores begin to compete for
consumers when the distance becomes sufficiently small. As a result, chains
face two opposing forces when making location choices: the chain effect and
the business stealing effect. It is conceivable that in some areas stores
are so close that the business stealing effect outweighs the gains and $%
\delta _{ii}$ becomes negative.

Holmes (2005) estimates that for places with a population density of 20,000
people per five-mile radius (which is comparable to an average city in my
sample counties), 89\% of the average consumers visits a Wal-Mart near by.%
\footnote{%
This is the result from a simulation exercise where the distance is set to 0
mile.} When the distance increases to 5 miles, 44\% of the consumers visits
the store. The percentage drops to 7\% if the store is 10 miles away. Survey
studies also show that few consumers drive further than 10-15 miles for
general merchandise shopping. In my sample, the median distance to the
nearest store is 21 miles for Wal-Mart stores, and 27 miles for Kmart
stores. It seems reasonable to think that the business stealing effect, if
it exists, is small.

\subsubsection{Independent error terms:\ $\protect\varepsilon _{m}$}

In the model, I\ have assumed that the market-level profit shocks $%
\varepsilon _{m}$ are independent across markets. Under this assumption, the
chain effect is identified from the geographic clustering pattern of the
store locations. Theoretically, one can use the number of small stores
across markets to identify the correlation in the error term, because small
stores are assumed to be single-unit firms. Conditioning on the covariates$,$
the number of small stores across markets is independent if there is no
cross-sectional dependence in the error terms. Once we control for the
cross-sectional dependence, the extra clustering exhibited by the chain
stores' location choice should be attributed to the chain effect. However,
implementing this idea is difficult as there is no easy way to simulate a
large number of error terms that exhibit dependence with irregular spatial
patterns. Therefore, cross-sectional dependence of the error term is
potentially another explanation for the spatial clustering pattern that I\
currently attribute to the chain effect.

\subsubsection{Extensions\label{Extension}}

Extending the model to allow for multiple stores in any given market
involves only a slight modification. In solving the best response given the
rival's strategy, instead of starting from $D=\{1,...,1\},$ we use $%
D=\{N_{1},...,N_{M}\},$ where $N_{m}$ is the maximum number of stores a
chain can potentially open in a given market $m$. The iteration will
converge within $\sum_{m}N_{m}$ steps. Notice that even though the size of
the strategy space has increased from $2^{M}$ to $\Pi _{m=1}^{M}N_{m}$, the
number of iterations only increases linearly, rather than exponentially, as
there are at most $\sum_{m}N_{m}$ steps for $D=\{N_{1},...,N_{M}\}$ to 
\textit{monotonically} decrease to $\{0,...,0\}.$ In general, the
computational complexity increases linearly with the number of stores in
each market. There is one caveat: when there are multiple stores in a market
that are owned by the same firm, the negative business stealing effect can
potentially outweigh the positive spillover effect. As a result, the
assumption that $\delta _{ii}\geq 0$ might not be supported by data in some
applications.

\section{Results}

\subsection{Parameter estimates\label{Parameter}}

The sample includes 2065 small- and medium-sized counties with populations
between 5,000 and 64,000 in the 1980s. Even though I do not model Kmart's
and Wal-Mart's entry decisions in other counties, I incorporate into the
profit function the spillover from stores outside the sample. This is
especially important for Wal-Mart, as the number of Wal-Mart stores in big
counties doubled over the sample period. Table III displays the summary
statistics of the distance weighted numbers of adjacent Kmart stores $\Sigma
_{l\neq m,l\in B_{m}}\frac{D_{k,l}}{Z_{ml}}$ and Wal-Mart stores $\Sigma
_{l\neq m,l\in B_{m}}\frac{D_{w,l}}{Z_{ml}}$, which measure the spillover
from nearby stores (including stores outside the sample). In 1997, the Kmart
spillover variable was slightly higher than in 1988 (0.13 vs. 0.11), but the
Wal-Mart spillover variable was almost twice as big as in 1988 (0.19 vs.
0.10).

The profit functions of all retailers share three common explanatory
variables: log of population, log of real retail sales per capita, and the
percentage of population that is urban. Many studies have found a pure size
effect:\ there tend to be more stores in a market as the population
increases. Retail sales per capita capture the \textquotedblleft
depth\textquotedblright\ of a market and explain firm entry behavior better
than personal income does. The percentage of urban population measures the
degree of urbanization. It is generally believed that urbanized areas have
more shopping districts that attract big chain stores.

For Kmart, the profit function includes a dummy variable for the Midwest
regions. Kmart's headquarters are located in Troy, Michigan. Until the mid
1980s, this region had always been the \textquotedblleft
backyard\textquotedblright\ of Kmart stores. Similarly, Wal-Mart's profit
function includes a southern dummy, as well as the log of distance in miles
to its headquarters in Bentonville, Arkansas. This distance variable turns
out to be a useful predictor for Wal-Mart stores' location choices. For
small firms, everything else equal, there are more small firms in the
southern states. It could be that there have always been fewer big retail
stores in the southern regions and that people rely on neighborhood small
firms for day-to-day shopping. The constant in the small firms' profit
function is allowed to differ between the \textquotedblleft
pre-chain\textquotedblright\ period and the \textquotedblleft
post-chain\textquotedblright\ period, to capture some general trend in the
number of small stores that is unrelated with chain stores' entry.

Table IV and V list the parameter estimates for the full model in six
different specifications. Table IV use the 1988 data for the
\textquotedblleft post-chain\textquotedblright\ period, while Table V use
the 1997 data for this period. The first five columns focus on the
competition between chains and small discount stores. The last column
estimates the model using Kmart, Wal-Mart, and all other discount stores,
including the small ones. The first column is the baseline specification,
where the estimates are obtained using the equilibrium most profitable for
Kmart. Column two estimates the model using the equilibrium most profitable
for Wal-Mart, while column three repeats the exercise using the equilibrium
that grants Wal-Mart an advantage in the south and Kmart an advantage in the
rest of the country. The estimates are quite similar across the different
equilibria.

One might be concerned that retail sales is endogenous:\ conditioning on
demographics, counties with a Kmart or Wal-Mart store will generate more
retail sales.\footnote{%
According to Kmart and Wal-Mart's annual report, the combined sales of Kmart
and Wal-Mart accounted for about 2\% of U.S. retail sales in 1988, and 4\%
in 1997. As I do not observe sales at the store level, I can not directly
measure how much a Kmart or a Wal-Mart store contributes to the total retail
sales in the counties where it is located. However, given that there are on
average 400-500 retailers per county, and that these two firms only
accounted for 2-4\% of total retail sales, the endogeneity of the retail
sales is not likely to be a severe problem.} In column 4, I\ estimated the
model using personal income per capita in place of the retail sales variable.%
\footnote{%
I did not use personal income per capita in small stores' profit function,
because it does not explain variations in the number of small stores very
well. In the OLS regression of the number of small stores on market size
variables, personal income per capita is not significant once population is
included.} Neither the competition effects not the chain effects change
much. The objective function value is higher, indicating a worse fit of the
data.

The model assumes that stores in different markets do not compete with each
other. However, it is possible that a chain store becomes a stronger
competitor when it is surrounded by a large number of stores owned by the
same firm in nearby markets. As a result, rival stores in neighboring
markets can indirectly affect competition between stores in a given market.
Column five estimates the following profit function for chain stores:%
\begin{equation*}
\left\{ 
\begin{array}{c}
\Pi _{i,m}(D_{i},D_{j,m},N_{s,m};~X_{m},Z_{m},\varepsilon _{m},\eta
_{i,m})=D_{i,m}\ast \lbrack X_{m}\beta _{i}+\delta _{ij}D_{j,m}\ast
(1+\delta _{ij,2}\Sigma _{l\neq m}\frac{D_{j,l}}{Z_{ml}})+ \\ 
\ \ \ \ \ \ \ \ \ \ \ \ \ \ \ \ \ \ \ \ \ \ \ +\delta _{is}\ln (N_{s,m}+1)+\
\delta _{ii}\Sigma _{l\neq m}\frac{D_{i,l}}{Z_{ml}}+\sqrt{1-\rho ^{2}}%
\varepsilon _{m}+\rho \eta _{i,m}],~i,j\in \{k,w\}%
\end{array}%
\right.
\end{equation*}%
where the competition effect $\delta _{ij}D_{j,m}$ is augmented by $\delta
_{ij,2}\Sigma _{l\neq m}\frac{D_{j,l}}{Z_{ml}},$ which is the distance
weighted number of rival stores in the nearby markets. Neither $\delta
_{kw,2}$ nor $\delta _{wk,2}$ is significant. The magnitude is also small:
on average, the competition effect is only raised by 2\%-3\% due to the
presence of surrounding stores.

In the rest of this section, I\ focus on the coefficients of the market size
variables $\beta $. I discuss the competition effects and the chain effects
in the next section.

The $\beta $ coefficients are highly significant and intuitive, with the
exception of the Midwest dummy, which is marginally significant in two
specifications in 1997. $\rho $ is smaller than $1$, indicating the
importance of the market level error terms and the necessity of controlling
for endogeneity of all firms' entry decisions.

Tables VI and VII display the model's goodness of fit for the baseline
specification.\footnote{%
The results for the rest of the specifications are available upon request.}
In Table VI, columns one and three display the sample averages, while the
other two columns list the model's predicted averages. The model matches
exactly the observed average numbers of Kmart and\ Wal-Mart stores. The
number of small firms is a noisy variable and is much harder to predict. Its
sample median is around 3 or 4, but the maximum is 20 in 1978, 25 in 1988,
and 19 in 1997. The model does a decent job of fitting the data. The sample
average is 4.75, 3.79, and 3.46 per county in 1978, 1988, and 1997,
respectively. The model's prediction is 4.80, 3.78, and 3.39, respectively.%
\footnote{%
I have estimated the three-stage model twice. In the first time, I use data
in 1978 for the \textquotedblleft pre-chain\textquotedblright\ period and
data in 1988 for the \textquotedblleft post-chain\textquotedblright\ period.
In the second time, I\ use data in 1978 and data in 1997 for the pre- and
post-chain\ period, respectively. Therefore, the model has two predictions
for the number of small stores in 1978, one from each estimation. In both
cases, the model's prediction comes very close to the sample mean.} Such
results might be expected as the parameters are chosen to match these
moments. In Table VII, I report the correlations between the predicted and
observed numbers of Kmart stores, Wal-Mart stores, and small firms in each
market. The numbers vary between 0.61 and 0.75. These correlations are not
included in the set of moment functions, and a high value indicates a good
fit. Overall, the model explains the data well.

To investigate the differences across equilibria, Table VIII reports the
percentage of markets where the two extreme equilibria differ. It turns out
that these equilibria are not very different from each other. For example,
in 1988, using the baseline parameter estimates, the equilibrium most
profitable for Kmart and the equilibrium most profitable for Wal-Mart differ
in only 1.41\% of the markets.\footnote{%
The numbers reported here are the average over 300 simulations.} As all
equilibria are bounded between these two extreme equilibria, the difference
between any pair of equilibria can only be (weakly) smaller.

In the absence of the chain effect, the only scenario that accommodates
multiple equilibria is when both a Kmart store and a Wal-Mart store are
profitable as the only chain store in the market, but neither is profitable
when both stores are in the market.\footnote{%
With the chain effect, all four cases -- (Kmart out, Wal-Mart out), (Kmart
out, Wal-Mart in), (Kmart in, Wal-Mart out), and (Kmart in, Wal-Mart in) --
can be the equilibrium outcome for a given market. Consider an example with
two markets. In market $A$, $\Pi _{k}^{A}=-0.2-0.6D_{w}^{A},$ $\Pi
_{w}^{A}=-0.2+0.3D_{w}^{B}-0.7D_{k}^{A};$ in market $B,$ $\Pi
_{k}^{B}=0.1-0.6D_{w}^{B},$ $\Pi _{w}^{B}=0.1+0.3D_{w}^{B}-0.7D_{k}^{B}.$
One can verify that there are two equilibria in this game. The first one is: 
$(D_{k}^{A}=0,D_{w}^{A}=0;~D_{k}^{B}=1,D_{w}^{B}=0),$ and the second one is: 
$(D_{k}^{A}=0,D_{w}^{A}=1;~D_{k}^{B}=0,D_{w}^{B}=1).$ In this simple
example, both (Kmart out, Wal-Mart out) and (Kmart out, Wal-Mart in) can be
the equilibrium outcome for market $A.$\label{NonUnique}} Accordingly, the
two possible equilibrium outcomes for a given market are: Kmart in and
Wal-Mart out, or Kmart out and Wal-Mart in.\footnote{%
The discussion on multiple equilibria has ignored the small stores, as the
number of small stores is a well-defined function of a given pair of (Kmart,
Wal-Mart).} Using the baseline parameter estimates, on average, this
situation arises in 1.1\% of the sample in 1988 and 1.4\% of the sample in
1997. These findings suggest that while multiple equilibria is potentially
an issue, it is not a prevalent phenomenon in the data.\footnote{%
In their study of banks' adoption of the automated clearinghouse electronic
payment system, Ackerberg and Gowrisankaran (2007) also found that the issue
of multiple equilibria is not economically significant.}$^{,}$\footnote{%
As one referee pointed out, multiple equilibria could potentially be more
important if the sample is not restricted to small- and medium- sized
counties. The exercise here has taken entry decisions and the benefit
derived from stores located in the metropolitan areas as given. It is
possible that multiple equilibria will occur more frequently once these
entry decisions are endogenized.} It also suggests that using different
profit functions for different firms helps to reduce the occurrence of
multiple equilibria in this entry model, because the more asymmetric firms
are in any given market, the less likely the event occurs where both firms
are profitable as the only chain store, but neither is profitable when both
operate in the market.

To understand the magnitudes of the market size coefficients, I report in
Table IX the changes in the number of each type of stores when some market
size variable changes using the estimates from the baseline specifications.%
\footnote{%
To save space, results from other specifications are not reported here. They
are not very different from those of the baseline specification.} For
example, to derive the effect of population change on the number of Kmart
stores, I\ fix Wal-Mart's and small stores' profits, increase Kmart's profit
in accordance with a ten percent increase in population, and re-solve the
full model to obtain the new equilibrium number for 300 simulations. For
each of these counter-factual exercises, the columns labeled with
\textquotedblleft Favors Kmart\textquotedblright\ use the equilibrium that
is most favorable for Kmart, while the columns labeled with
\textquotedblleft Favors Wal-Mart\textquotedblright\ uses the other extreme
equilibrium. They provide an upper (lower) and lower (upper) bound for the
number of Kmart (Wal-Mart) stores. It should not come as a surprise that
results of these two scenarios are quite similar, since the two equilibria
are not very different. In the following discussion, I focus on the
equilibrium most profitable for Kmart.

Market size variables are important for big chains. In 1988, a 10\% growth
in population induces Kmart to enter 10.5\% more markets and Wal-Mart 8.6\%
more markets. A similar increment in retail sales attracts entry of Kmart
and Wal-Mart stores in 16.8\% and 10.3\% more markets, respectively. The
results are similar for 1997. These differences indicate that Kmart is much
more likely to locate in bigger markets, while Wal-Mart thrives in smaller
markets. Perhaps not surprisingly, the regional advantage is substantial for
both chains: controlling for the market size, changing the Midwest regional
dummy from 1 to 0 for all counties leads to 33.1\% fewer Kmart stores, and
changing the Southern regional dummy from 1 to 0 for all counties leads to
53.2\% fewer Wal-Mart stores. When distance increases by 10\%, the number of
Wal-Mart stores drops by 8.8\%. Wal-Mart's \textquotedblleft home
advantage\textquotedblright\ is much smaller in 1997: everything else the
same, changing the south dummy from 1 to 0 for all counties leads to 29\%
fewer Wal-Mart stores, and a 10\% increase in distance reduces the number of
Wal-Mart stores by only 4\%. As the model is static in nature\ (all Kmart
and Wal-Mart stores are opened in one period), the regional dummies and the
distance variable provide a reduced-form way to capture the path-dependence
of the expansion of chain stores.

The market size variables have a relatively modest impact on the number of
small businesses. In 1988, a 10\% increase in population attracts 6.6\% more
stores. The same increase in real retail sales per capita draws 4.9\% more
stores. The number of small stores declines by about 1.8\% when the
percentage of urban population goes up by 10\%. In comparison, the regional
dummy is much more important: everything else equal, changing the southern
dummy from 1 to 0 for all counties leads to 33.3\% fewer small stores (6290
stores vs. 9431 stores). When the sunk cost increases by 10\%, the number of
small stores reduces by 4.1\%.

\subsection{The competition effect and the chain effect}

As shown in Table VI and V, all competition effects in the profit function
of the small stores and that of all other discount stores are precisely
estimated. The chain effect and the competition effect in Wal-Mart's profit
function are also reasonably well estimated. The results for Kmart's profit
function appear to be the weakest: although the size of the coefficients are
similar, the standard errors are large for some columns. For example, the
chain effect is significant in 4 out of 6 specifications in 1988, and in
only two specifications in 1997. The competition effect of Wal-Mart on Kmart
is big and significant in all cases in 1997, but insignificant in two
specifications in 1988. The impact of small stores on the chain stores are
never very significant. With one exception in 1997, both $\tau $ and the
sunk cost are significant and sizeable, indicating the importance of history
dependence.

To assess the magnitude of the competition effects for the chains, Table XII
re-solve the equilibrium number of Kmart and Wal-Mart stores under different
assumptions of the market structure. The negative impact of Kmart's presence
on Wal-Mart's profit is much stronger in 1988 than in 1997, while the
opposite is true for the effect of Wal-Mart's presence on Kmart's profit.
For example, in 1988, Wal-Mart would only enter 400 markets if there were a
Kmart store in every county. When Kmart ceases to exist as a competitor, the
number of markets with Wal-Mart stores rises to 778, a net increase of
94.5\%. The same experiment in 1997 leads Wal-Mart to enter 28.8\% more
markets, from 809 to 1042. The pattern is reversed for Kmart. In 1988, Kmart
would enter 27.8\% more markets when there are no Wal-Mart stores compared
with the case of one Wal-Mart store in every county (474 Kmart stores vs.
371 Kmart stores); in 1997, Kmart would enter 72.8\% more markets for the
same experiment (558 Kmart stores vs. 323 Kmart stores).\footnote{%
In solving for the number of Wal-Mart (Kmart) stores when Kmart (Wal-Mart)
exits, I\ allow the small firms to compete with the remaining chain.} These
estimates are consistent with the observation that Wal-Mart grew stronger
during the sample period and replaced Kmart as the largest discounter in
1991.

Both a Cournot model and a Bertrand model with differentiated products
predict that reduction in rivals' marginal costs drives down a firm's own
profit. I do not observe firms' marginal costs, but these estimates are
consistent with evidence that Wal-Mart's marginal cost was declining
relative to Kmart's over the sample period. Wal-Mart is famous for its
cost-sensitive culture; it is also keen on technology advancement. Holmes
(2001) cites evidence that Wal-Mart has been a leading investor in
information technology. In contrast, Kmart struggled with its management
failures that resulted in stagnant revenue sales, and it either delayed or
abandoned store renovation plans throughout the 1990s.

To investigate the importance of the chain effect for both chains, the last
row of both panels in Table XII reports the equilibrium number of stores
when there is no chain effect. I set $\delta _{ii}=0$ for the targeted
chain, but keep the rival's $\delta _{jj}$ unchanged and re-solve the model.
The difference in the number of stores with or without $\delta _{ii}$
captures the advantage of chains over single-unit retailers. In 1988,
without the chain effect, the number of Kmart stores would have decreased by
5.3\%, and Wal-Mart would have entered 15.6\% fewer markets. In 1997, Kmart
would have entered 6.5\% fewer markets while Wal-Mart 7.1\%. The decline in
Wal-Mart's chain effect suggests that the benefit of scale economies does
not grow proportionally. In fact there are good reasons to believe it might
not be monotone because, as discussed in section \ref{Delta}, when chains
grow bigger and saturate the area, cannibalization among stores becomes a
stronger concern.

As I do not observe the stores' sales or profit, I cannot estimate the
dollar value of these spillover benefits. However, given the low markup of
these discount stores (the average gross markup was 20.9\% from 1993 to
1997, see footnote 2), these estimates appear to be large. The results are
consistent with Holmes (2005), who also found scale economies to be
important. Given the magnitude of these spillover effects, further research
that explains their mechanism will help improve our understanding of the
retail industry, in particular its productivity gains over the past several
decades.\footnote{%
See Foster \textit{et al} (2002) for a detailed study of the productivity
growth in the retail industry.}

Table XIII\ studies the competition effects of\ chains on small discount
stores. Here I distinguish between two cases. The first two columns report
the number of small stores predicted by the model, where small stores
continue their business after the entry of Kmart and Wal-Mart as long as
their profit is positive, even if they cannot recover the sunk cost paid in
the first stage. The second two columns report the number of small stores
whose \textquotedblleft post-chain\textquotedblright\ profit is higher than
the sunk cost. If small stores had perfect foresight and could predict the
entry of Kmart and Wal-Mart, these two columns would be the number of stores
that we observe. The results suggest that chains have a substantial
competition impact on small firms. In 1988, compared with the scenario with
no chain stores, adding a Kmart store to each market reduces the number of
small firms by 23.8\%, or 1.07 stores per county. Of the remaining stores,
more than one-third could not recover their sunk cost of entry. Had they
learned of the entry of the chains stores in the first stage, they would not
have entered the market. Thus, adding a Kmart store makes 52.1\% of the
small stores, or 2.33 stores per county, either unprofitable or unable to
recover their sunk cost. The story is similar for the entry of Wal-Mart
stores. When both a Kmart and a Wal-Mart store enter, 68.4\% of the small
stores, or 3.07 stores per county, cannot recoup their sunk cost of entry.

Looking at the discount industry as a whole, the impact of Kmart and
Wal-Mart remains significant, although Kmart's impact is slightly diminished
in 1997. Table XIV shows that when a Wal-Mart store enters a market in 1988,
21.5\% of the discount firms will exit the market, and 56.4\% of the firms
cannot recover their sunk cost. These numbers translate to 1.1 stores and
2.9 stores, respectively.

It is somewhat surprising that the negative impact of Kmart on other firms'
profit is comparable to Wal-Mart's impact, considering the controversies and
media reports generated by Wal-Mart. The outcry about Wal-Mart was probably
because Wal-Mart had more stores in small- to medium-sized markets where the
effect of a big store entry was felt more acutely, and because Wal-Mart kept
expanding, while\ Kmart was consolidating its existing stores with few net
openings in these markets over the sample period.

\subsection{The impact of Wal-Mart's expansion and related policy issues 
\label{Wal-Mart}}

Consistent with media reports about Wal-Mart's impact on small retailers,
the model predicts that Wal-Mart's expansion contributes to a large
percentage of the net decline in the number of small firms over the sample
period. The first row in Table XV records the net decrease of 693 small
firms observed over the sample period, or 0.34 per market. To evaluate the
impact of Wal-Mart's expansion on small firms separately from other factors
(e.g., the change in market sizes or the change in Kmart stores), I re-solve
the model using the 1988 coefficients and the 1988 market size variables for
Kmart's and small firms' profit functions, but the 1997 coefficients and
1997 market size variables for Wal-Mart's profit function. The experiment
corresponds to holding small stores and Kmart the same as in 1988, but
allowing Wal-Mart to become more efficient and expand. The predicted number
of small firms falls by 380. This accounts for 55\% of the observed decrease
in the number of small firms. Conducting the same experiment but using the
1997 coefficients and the 1997 market size variables for Kmart's and small
firms' profit functions, and the 1988 coefficients and 1988 market size
variables for Wal-Mart's profit function, I find that Wal-Mart's expansion
accounts for 259 stores, or 37\% of the observed decrease in the number of
small firms.

Repeating the same exercise using all discount stores, the prediction is
similar: roughly thirty to forty percent of store exits can be attributed to
the expansion of Wal-Mart stores. Overall, the absolute impact of Wal-Mart's
entry seems modest. However, the exercise here only looks at firms in the
discount sector. Both Kmart and Wal-Mart carry a large assortment of
products and compete with a variety of stores, like hardware stores,
houseware stores, apparel stores, etc., so that their impact on local
communities is conceivably much larger.

I tried various specifications that group retailers in different sectors,
for example, all retailers in the discount sector, the building materials
sector, and the home-furnishing sector. None of these experiments was
successful, as the retailers in different sectors differ substantially and
the simple model cannot match the data very well. Perhaps a better approach
is to use a separate profit function for firms in each sector and estimate
the system of profit functions jointly. This is beyond the scope of this
paper and is left for future research.

Government subsidy has long been a policy instrument to encourage firm
investment and to create jobs. To evaluate the effectiveness of this policy
in the discount retailing sector, I simulate the equilibrium numbers of
stores when various firms are subsidized. The results in Table XVI indicate
that direct subsidies do not seem to be effective in generating jobs. In
1988, subsidizing Wal-Mart stores 10\% of their average profit increases the
number of Wal-Mart stores per county from 0.32 to 0.34.\footnote{%
The average Wal-Mart store's net income in 1988 is about one million in 2004
dollars according its SEC annual report. Using a discount rate of 10\%, the
discounted present value of a store's lifetime profit is about ten million.
A subsidy of 10\% is roughly one million dollars.}$^{,}$\footnote{%
In this exercise, I first simulate the model 300 times, obtain the mean
profit for all Wal-Mart stores for each simulation, and average it across
simulations. Then I increase Wal-Mart's profit by 10\% of this average (that
is, I add this number to the constant of Wal-Mart's profit function), and
simulate the model 300 times to obtain the number of Wal-Mart stores after
the subsidy.} With the average Wal-Mart store hiring fewer than 300 full and
part-time employees, the additional number of stores translates to roughly
seven new jobs. Wal-Mart's expansion crowds out other stores, which brings
the net increase down to six jobs. Similarly, subsidizing all small firms by
100\% of their average profit increases their number from 3.78 to 5.07, and
generates thirteen jobs if on average a small firm hires ten employees.
Repeating the exercise with subsidizing all discount stores (except for
Kmart and Wal-Mart stores) by 100\% of their average profit leads to a net
increase of thirty-four jobs. Together, these exercises suggest that a
direct subsidy does not seem to be very effective in generating employment
in this industry. These results reinforce the concerns raised by many policy
observers regarding the subsidies directed to big retail corporations.
Perhaps less obvious is the conclusion that subsidies toward small retailers
should also be designed carefully.

\section{Conclusion and future work}

In this paper, I have examined the competition effect between Kmart stores,
Wal-Mart stores, and other discount stores, as well as the role of the chain
effect in firms' entry decisions. The negative impact of Kmart's presence on
Wal-Mart's profit is much stronger in 1988 than in 1997, while the opposite
is true for the effect of Wal-Mart's presence on Kmart's profit. On average,
entry by either a Kmart or a Wal-Mart store makes 48\% to 58\% of the
discount stores, or two to three stores, either unprofitable or unable to
recover their sunk cost. Wal-Mart's expansion from the late 1980s to the
late 1990s explains 37\% to 55\% of the net change in the number of small
discount stores, and 34\% to 41\% of the net change in the number of all
discount stores.

Like Holmes (2005), I find that scale economies, as captured by the chain
effect, generate substantial benefits. Without the spillover effect, the
number of Kmart stores would have decreased by 5.3\% in 1988 and 6.5\% in
1997, while Wal-Mart would have entered 15.6\% fewer markets in 1988 and
7.1\% fewer markets in 1997. Studying these scale economies in more detail
is useful for guiding merger policies or other regulations that affect
chains. A better understanding of the mechanism underlying these spillover
effects will also help us to gain insights in the productivity gains in the
retail industry over the past several decades.

Finally, the algorithm used in this paper can be applied to industries where
scale economies are important. One possible application is to industries
with cost complementarity among different products. The algorithm here is
particularly suitable for modeling firms' product choices when the product
space is large.

\section*{Appendix A: data}

I went through all the painstaking details to clean the data from the
Directory of Discount Stores. After the manually entered data were inspected
many times with the hard copy, the stores' cities were matched to belonging
counties using a census data.\footnote{%
Marie Pees from the census bureau kindly provided these data.} Some city
names listed in the directory contained typos, so I first found possible
spellings using the census data, then inspected the stores' street addresses
and zipcodes using various web sources to confirm the right city name
spelling. The final data set appears to be quite accurate. I compared it
with Wal-Mart's firm data and found the difference to be quite small.%
\footnote{%
I am very grateful to Emek Basker for sharing the Wal-Mart firm data with me.%
} For the sample counties, only thirty to sixty stores were not matched
between these two sources for either 1988 or 1997.

\section*{Appendix B: Definitions and proofs}

\subsection*{B.1 \ Verification of the necessary condition (\protect\ref%
{Nece}) \label{NeceApp}}

Let $D^{\ast }=\arg \max_{D\in \mathbf{D}}\Pi (D)$. The optimality of $%
D^{\ast }$ implies the following set of necessary conditions:%
\begin{equation*}
\Pi (D_{1}^{\ast },...,D_{m-1}^{\ast },D_{m}^{\ast },D_{m+1}^{\ast
},...,D_{M}^{\ast })\geq \ \Pi (D_{1}^{\ast },...,D_{m-1}^{\ast
},D_{m},D_{m+1}^{\ast },...,D_{M}^{\ast }),\forall m,D_{m}^{\ast }\neq D_{m}
\end{equation*}%
Let $\hat{D}=\{D_{1}^{\ast },...,D_{m-1}^{\ast },D_{m},D_{m+1}^{\ast
},...,D_{M}^{\ast }\}.$ $\Pi (D^{\ast })$ differs from $\Pi (\hat{D})$ in
two parts: the profit in market $m,$ and the profit in all other markets
through the chain effect:%
\begin{eqnarray*}
\Pi (D^{\ast })-\Pi (\hat{D}) &=&(D_{m}^{\ast }-D_{m})[X_{m}+\delta \Sigma
_{l\neq m}\frac{D_{l}^{\ast }}{Z_{ml}}]+ \\
&&\delta \Sigma _{l\neq m}D_{l}^{\ast }(\frac{D_{m}^{\ast }}{Z_{lm}})-\delta
\Sigma _{l\neq m}D_{l}^{\ast }(\frac{D_{m}}{Z_{lm}}) \\
&=&(D_{m}^{\ast }-D_{m})[X_{m}+2\delta \Sigma _{l\neq m}\frac{D_{l}^{\ast }}{%
Z_{ml}}]
\end{eqnarray*}%
where $Z_{ml}=Z_{lm}$ due to symmetry. Since $\Pi (D^{\ast })-\Pi (\hat{D}%
)\geq 0,$ $D_{m}^{\ast }\neq D_{m}$, it must be that: $D_{m}^{\ast
}=1,D_{m}=0$ if and only if $X_{m}+2\delta \Sigma _{l\neq m}\frac{%
D_{l}^{\ast }}{Z_{ml}}\geq 0;$ and $D_{m}^{\ast }=0,D_{m}=1$ if and only if $%
X_{m}+2\delta \Sigma _{l\neq m}\frac{D_{l}^{\ast }}{Z_{ml}}<0.$ Together we
have: $D_{m}^{\ast }=1[X_{m}+2\delta \Sigma _{l\neq m}\frac{D_{l}^{\ast }}{%
Z_{ml}}\geq 0]$.\footnote{%
I have implicitly assumed that when $X_{m}+2\delta \Sigma _{l\neq m}\frac{%
D_{l}^{\ast }}{Z_{ml}}=0,$ $D_{m}^{\ast }=1.$}

\subsection*{B.2 \ The set of fixed points of an increasing function that
maps a lattice into itself}

Tarski's fixed point theorem, stated in the main body of the paper as
Theorem \ref{Tarski}, establishes that the set of fixed points of an
increasing function that maps from a lattice into itself is a nonempty
complete lattice with a greatest element and a least element. For a
counterexample where a decreasing function's set of fixed points is empty,
consider the following simplified entry model where three firms compete with
each other and decide simultaneously whether to enter the market. The profit
functions are as follows:%
\begin{equation*}
\left\{ 
\begin{array}{c}
\Pi _{k}=D_{k}(0.5-D_{w}-0.25D_{s}) \\ 
\Pi _{w}=D_{w}(1-0.5D_{k}-1.1D_{s}) \\ 
\Pi _{s}=D_{s}(0.6-0.7D_{k}-0.5D_{w})%
\end{array}%
\right.
\end{equation*}

Let $D=\{D_{k},D_{w},D_{s}\}\in \mathbf{D=\{0,1\}}^{3},~D_{-i}$ denote
rivals' strategies, $V_{i}(D_{-i})$ denote the best response function for
player $i$, and $V(D)=\{V_{k}(D_{-k}),V_{w}(D_{-w}),V_{s}(D_{-s})\}$ denote
the joint best response function. It is easy to show that $V(D)$ is a
decreasing function that takes the following values:%
\begin{equation*}
\left\{ 
\begin{array}{c}
V(0,0,0)=\{1,1,1\};~V(0,0,1)=\{1,0,1\};~V(0,1,0)=\{0,1,1\};~V(0,1,1)=\{0,0,1%
\} \\ 
V(1,0,0)=\{1,1,0\};~V(1,0,1)=\{1,0,0\};~V(1,1,0)=\{0,1,0\};~V(1,1,1)=\{0,0,0%
\}%
\end{array}%
\right.
\end{equation*}%
The set of fixed points of $V(D)$ is empty$.$

\subsection*{B.3 \ A tighter lower bound and upper bound for the optimal
solution vector $D^{\ast }$}

In section \ref{OneAgent} I have shown that using $\inf (\mathbf{D})$ and $%
\sup (\mathbf{D})$ as starting points yields, respectively, a lower bound
and an upper bound to $D^{\ast }=\arg \max_{D\in \mathbf{D}}\Pi (D).$ Here
I\ introduce two bounds that are tighter. The lower bound builds on the
solution to a constrained maximization problem:%
\begin{eqnarray*}
\max\limits_{D_{1,}...,D_{M}\in \{0,1\}}\Pi &=&\sum_{i=1}^{M}\left[
D_{m}\ast (X_{m}+\delta \Sigma _{l\neq m}\frac{D_{l}}{Z_{ml}})\right] \\
s.t.\text{ if }D_{m} &=&1,\text{ then }X_{m}+\delta \Sigma _{l\neq m}\frac{%
D_{l}}{Z_{ml}}>0
\end{eqnarray*}%
The solution to this constrained maximization problem belongs to the set of
fixed points of the vector function $\hat{V}(D)=\{\hat{V}_{1}(D),...,\hat{V}%
_{M}(D)\}$, where $\hat{V}_{m}(D)=1[X_{m}+\delta \Sigma _{l\neq m}\frac{D_{l}%
}{Z_{ml}}>0].$ When $\delta >0,$ the function $\hat{V}(\cdot )$ is
increasing and maps from $\mathbf{D}$ into itself: $\hat{V}:\mathbf{D}%
\rightarrow \mathbf{D}.$ Let $\hat{D}$ denote the convergent vector using $%
\sup (\mathbf{D})$ as the starting point for the iteration on $\hat{V}:\hat{V%
}(\hat{D})=\hat{D}$. Using arguments similar to those in section \ref%
{OneAgent}, one can show that $\hat{D}$ is the greatest element among the
set of $\hat{V}$'s fixed points. Further, $\hat{D}$ achieves a higher profit
than any other fixed point of $\hat{V}(\cdot )$, since by construction each
non-zero element of the vector $\hat{D}$ adds to the total profit$.$
Changing any non-zero element(s) of $\hat{D}$ to zero reduces the total
profit.

To show that $\hat{D}\leq D^{\ast },$ the solution to the original
unconstrained maximization problem, we construct a contradiction. Since the
maximum of an unconstrained problem is always greater than that of a
corresponding constrained problem, we have: $\Pi (D^{\ast })\geq \Pi (\hat{D}%
)$. Therefore, $D^{\ast }$ can't be strictly smaller than $\hat{D},$ because
any vector strictly smaller than $\hat{D}$ delivers a lower profit. Suppose $%
D^{\ast }$ and $\hat{D}$ are unordered. Let $D^{\ast \ast }=D^{\ast }\vee 
\hat{D}$ (where \textquotedblleft $\vee $\textquotedblright\ defines the
element-by-element Max operation). The change from $D^{\ast }$ to $D^{\ast
\ast }$ increases total profit, because profit at markets with $D_{m}^{\ast
}=1$ does not decrease after the change, and profit at markets with $%
D_{m}^{\ast }=0$ but $\hat{D}_{m}=1$ is positive by construction. This
contradicts the definition of $D^{\ast }$, so $\hat{D}\leq D^{\ast }.$

Note that $V(\hat{D})\geq \hat{V}(\hat{D})=\hat{D},$ where $V(\cdot )$ is
defined in section \ref{OneAgent}. As in section \ref{OneAgent}, iterating $%
V $ on both sides of the inequality $V(\hat{D})\geq \hat{D}$ generates an
increasing sequence. Denote the convergent vector as $\hat{D}^{T}.~$This is
a tighter lower bound of $D^{\ast }$ than $D^{L}$ (discussed in section \ref%
{OneAgent}) because $\hat{D}^{T}=V^{TT}(\hat{D})\geq V^{TT}(\inf (\mathbf{D}%
))=D^{L},$ with $TT=\max \{T,T^{\prime }\},$ where $T$ is the number of
iterations from $\hat{D}$ to $\hat{D}^{T}$ and $T^{\prime }$ is the number
of iterations from $\inf (\mathbf{D})$ to $D^{L}.$

Since the chain effect is bounded by zero and $\delta \Sigma _{l\neq m}\frac{%
1}{Z_{ml}}$, it is never optimal to enter markets that contribute a negative
element to the total profit even with the largest conceivable chain effect.
Let $\tilde{D}=\{\tilde{D}_{m}:\tilde{D}_{m}=0$ if $X_{m}+2\delta \Sigma
_{l\neq m}\frac{1}{Z_{ml}}<0;\tilde{D}_{m}=1$ otherwise$\}$. We know that $%
\tilde{D}\geq D^{\ast }.$ Using the argument above, the convergent vector $%
\tilde{D}^{T}$ from iterating $V$ on $\tilde{D}$ is a tighter upper bound to 
$D^{\ast }$ than $D^{U}$.

\subsection*{B.4 \ Verification that the chains' profit functions are
supermodular with decreasing differences}

\begin{definition}
Suppose that $Y(X)$ is a real-valued function on a lattice $\mathbf{X}.$ If%
\begin{equation}
Y(X^{\prime })+Y(X^{\prime \prime })\leq Y(X^{\prime }\vee X^{\prime \prime
})+Y(X^{\prime }\wedge X^{\prime \prime })  \label{Super}
\end{equation}%
for all $X^{\prime }$ and $X^{\prime \prime }$ in $\mathbf{X}$, then $Y(X)$
is supermodular on $\mathbf{X}$.\footnote{%
Both definitions are taken from Chapter 2 of Topkis (1998).}
\end{definition}

\begin{definition}
Suppose that $\mathbf{X}$ and $K$ are partially ordered sets and $Y(X,k)$ is
a real-valued function on $\mathbf{X}\times K.$ If $Y(X,k^{\prime \prime
})-Y(X,k^{\prime })$ is increasing, decreasing, strictly increasing, or
strictly decreasing in $X$ on $\mathbf{X}$ for all $k^{\prime }\prec
k^{\prime \prime }$ in $K$, then $Y(X,k)$ has, respectively, increasing
differences, decreasing differences, strictly increasing differences, or
strictly decreasing differences in $(X,k)$ on $\mathbf{X}.$
\end{definition}

Now let us verify that chain $i$'s profit function in the equation system (%
\ref{Profit}) is supermodular in its own strategy $D_{i}\in \mathbf{D}$. For
ease of notation, the firm subscript $i$ is omitted, and $X_{m}\beta
_{i}+\delta _{ij}D_{j,m}+\delta _{is}\ln (N_{s,m}+1)+\sqrt{1-\rho ^{2}}%
\varepsilon _{m}+\rho \eta _{i,m}$ is absorbed into $X_{m}$. The profit
function is simplified to: $\Pi =\sum_{m=1}^{M}\left[ D_{m}\ast
(X_{m}+\delta \Sigma _{l\neq m}\frac{D_{l}}{Z_{ml}})\right] .$ First it is
easy to show that $D^{\prime }\vee D^{\prime \prime }=(D^{\prime }-\min
(D^{\prime },D^{\prime \prime }))+(D^{\prime \prime }-\min (D^{\prime
},D^{\prime \prime }))+\min (D^{\prime },D^{\prime \prime }),$ and $%
D^{\prime }\wedge D^{\prime \prime }=\min (D^{\prime },D^{\prime \prime }).$
Let $D^{\prime }-\min (D^{\prime },D^{\prime \prime })$ be denoted as $%
D_{1},~D^{\prime \prime }-\min (D^{\prime },D^{\prime \prime })$ as $D_{2},$
and $\min (D^{\prime },D^{\prime \prime })$ as $D_{3}.$The left-hand side of
the inequality (\ref{Super}) is:%
\begin{eqnarray*}
\Pi (D^{\prime })+\Pi (D^{\prime \prime }) &=&\sum\nolimits_{m}D_{m}^{\prime
}(X_{m}+\delta \Sigma _{l\neq m}\frac{D_{l}^{\prime }}{Z_{ml}}%
)+\sum\nolimits_{m}D_{m}^{\prime \prime }(X_{m}+\delta \Sigma _{l\neq m}%
\frac{D_{l}^{\prime \prime }}{Z_{ml}}) \\
&=&\sum\nolimits_{m}\left[ (D_{m}^{\prime }-\min (D_{m}^{\prime
},D_{m}^{\prime \prime }))+\min (D_{m}^{\prime },D_{m}^{\prime \prime })%
\right] \ast \\
&&\left[ X_{m}+\delta \Sigma _{l\neq m}\frac{1}{Z_{ml}}[(D_{l}^{\prime
}-\min (D_{l}^{\prime },D_{l}^{\prime \prime }))+\min (D_{l}^{\prime
},D_{l}^{\prime \prime })]\right] + \\
&&\sum\nolimits_{m}[(D_{m}^{\prime \prime }-\min (D_{m}^{\prime
},D_{m}^{\prime \prime }))+\min (D_{m}^{\prime },D_{m}^{\prime \prime })]\ast
\\
&&\left[ X_{m}+\delta \Sigma _{l\neq m}\frac{1}{Z_{ml}}[(D_{l}^{\prime
\prime }-\min (D_{l}^{\prime },D_{l}^{\prime \prime }))+\min (D_{l}^{\prime
},D_{l}^{\prime \prime })]\right] \\
&=&\sum\nolimits_{m}(D_{1,m}+D_{3,m})(X_{m}+\delta \Sigma _{l\neq m}\frac{1}{%
Z_{ml}}(D_{1,l}+D_{3,l}))+ \\
&&\sum\nolimits_{m}(D_{2,m}+D_{3,m})(X_{m}+\delta \Sigma _{l\neq m}\frac{1}{%
Z_{ml}}(D_{2,l}+D_{3,l}))
\end{eqnarray*}

Similarly, the right-hand side of the inequality (\ref{Super}) is:%
\begin{eqnarray*}
\Pi (D^{\prime }\vee D^{\prime \prime })+\Pi (D^{\prime }\wedge D^{\prime
\prime }) &=&\sum\nolimits_{m}(D_{m}^{\prime }\vee D_{m}^{\prime \prime }) 
\left[ X_{m}+\delta \Sigma _{l\neq m}\frac{1}{Z_{ml}}(D_{l}^{\prime }\vee
D_{l}^{\prime \prime })\right] + \\
&&\sum\nolimits_{m}(D_{m}^{\prime }\wedge D_{m}^{\prime \prime })\left[
X_{m}+\delta \Sigma _{l\neq m}\frac{1}{Z_{ml}}(D_{l}^{\prime }\wedge
D_{l}^{\prime \prime })\right] \\
&=&\sum\nolimits_{m}(D_{1,m}+D_{2,m}+D_{3,m})\left[ X_{m}+\delta \Sigma
_{l\neq m}\frac{1}{Z_{ml}}(D_{1,l}+D_{2,l}+D_{3,l})\right] + \\
&&\sum\nolimits_{m}D_{3,m}(X_{m}+\delta \Sigma _{l\neq m}\frac{1}{Z_{ml}}%
D_{3,l}) \\
&=&\Pi (D^{\prime })+\Pi (D^{\prime \prime })+\delta (\Sigma _{m}\Sigma
_{l\neq m}\frac{D_{2,m}D_{1,l}+D_{1,m}D_{2,l}}{Z_{ml}})
\end{eqnarray*}%
The profit function is supermodular in its own strategy if the chain effect $%
\delta $ is non-negative. To verify that the profit function $\Pi _{i}$ has
decreasing differences in $(D_{i},D_{j})$:%
\begin{eqnarray*}
\Pi _{i}(D_{i},D_{j}^{\prime \prime })-\Pi _{i}(D_{i},D_{j}^{\prime })
&=&\sum\nolimits_{m}\left[ D_{i,m}\ast (X_{im}+\delta _{ii}\Sigma _{l\neq m}%
\frac{D_{i,l}}{Z_{ml}}+\delta _{ij}D_{j,m}^{\prime \prime })\right] - \\
&&\sum\nolimits_{m}\left[ D_{i,m}\ast (X_{im}+\delta _{ii}\Sigma _{l\neq m}%
\frac{D_{i,l}}{Z_{ml}}+\delta _{ij}D_{j,m}^{\prime })\right] \\
&=&\delta _{ij}\sum_{m=1}^{M}D_{i,m}(D_{j,m}^{\prime \prime
}-D_{j,m}^{\prime })
\end{eqnarray*}%
The difference is decreasing in $D_{i}$ for all $D_{j}^{\prime
}<D_{j}^{\prime \prime }$ as long as $\delta _{ij}\leq 0.$

\subsection*{B.5 \ Computational issues}

The main computational burden of this exercise is the search for the best
responses $K(D_{w})$ and $W(D_{k}).$ In section \ref{OneAgent}, I have
proposed two bounds $D^{U}$ and $D^{L}$ that help to reduce the number of
profit evaluations. Appendix B.3 illustrates a tighter upper bound and lower
bound that work well in the empirical implementation.

When the chain effect $\delta _{ii}$ is sufficiently big, it is conceivable
that the upper bound and lower bound are far apart from each other. If this
happens, computational burden once again becomes an issue, as there will be
many vectors between these two bounds.

Two observations work in favor of the algorithm. First, recall that the
chain effect is assumed to take place among counties whose centroids are
within fifty miles. Markets that are farther away are not directly
\textquotedblleft connected\textquotedblright : conditioning on the entry
decisions in other markets, the entry decisions in group $A$ do not depend
on the entry decisions in group $B$ if all markets in group $A$ are at least
fifty miles away from any market in group $B.$ Therefore, what matters is
the size of the largest connected markets different between $D^{U}$ and $%
D^{L},$ rather than the total number of elements different between $D^{U}$
and $D^{L}.$ To illustrate this point, suppose there are ten markets as
below:

$%
\begin{tabular}{l|l|l|ll}
\cline{2-4}
& 1 & 2 & 3 & \multicolumn{1}{|l}{} \\ \hline
\multicolumn{1}{|l|}{4} & 5 & 6 & 7 & \multicolumn{1}{|l|}{8} \\ \hline
& 9 & 10 &  &  \\ \cline{2-3}
\end{tabular}%
,$ where $D^{U}=%
\begin{tabular}{l|l|l|ll}
\cline{2-4}
& 1 & $D_{2}$ & 1 & \multicolumn{1}{|l}{} \\ \hline
\multicolumn{1}{|l|}{1} & 1 & $D_{6}$ & 1 & \multicolumn{1}{|l|}{1} \\ \hline
& $D_{9}$ & $D_{10}$ &  &  \\ \cline{2-3}
\end{tabular}%
,$ and$~D^{L}=$ $%
\begin{tabular}{l|l|l|ll}
\cline{2-4}
& 0 & $D_{2}$ & 0 & \multicolumn{1}{|l}{} \\ \hline
\multicolumn{1}{|l|}{0} & 0 & $D_{6}$ & 0 & \multicolumn{1}{|l|}{0} \\ \hline
& $D_{9}$ & $D_{10}$ &  &  \\ \cline{2-3}
\end{tabular}%
$. $D^{U}$ and $D^{L}$ are the same in markets 2, 6, 9, and 10, but differ
for the rest. If markets 1, 4, and 5 (group $A$) are at least fifty miles
away from markets 3, 7, and 8 (group $B$), one only needs to evaluate $%
2^{3}+2^{3}=16$ vectors, rather than $2^{6}=64$ vectors to find the profit
maximizing vector.

The second observation is that even with a sizable chain effect, the event
of having $D^{U}$ and $D^{L}$ different in a large connected area is
extremely unlikely. Let $N$ denote the size of such an area $C_{N}.$ Let $%
\xi _{m}$ denote the random shocks in the profit function. By construction, $%
D_{m}^{U}=1[X_{m}+2\delta \Sigma _{l\neq m,l\in B_{m}}\frac{D_{l}^{U}}{Z_{ml}%
}+\xi _{m}\geq 0]$ and $D_{m}^{L}=1[X_{m}+2\delta \Sigma _{l\neq m,l\in
B_{m}}\frac{D_{l}^{L}}{Z_{ml}}+\xi _{m}\geq 0].~$The probability of $%
D_{m}^{U}=1,D_{m}^{L}=0$ for every market in the size-$N$ connected area $%
C_{N}$ is:%
\begin{equation*}
\Pr (D_{m}^{U}=1,D_{m}^{L}=0,\forall m\in C_{N})\leq \Pi _{m=1}^{N}\Pr
(X_{m}+\xi _{m}<0,X_{m}+\xi _{m}+2\delta \Sigma _{l\neq m,l\in B_{m}}\frac{1%
}{Z_{ml}}\geq 0)
\end{equation*}%
where $\xi _{m}$ is assumed to be i.i.d. across markets$.$\ As $\delta $
goes to infinity, the probability approaches $\Pi _{m=1}^{N}\Pr (X_{m}+\xi
_{m}<0)$ from below. How fast it decreases when $N$ increases depends on the
distribution assumption$.$ If $\xi _{m}$ is i.i.d. normal, $X_{m}$ is i.i.d.
uniformly distributed between $[-a,a],$ with $a$ a finite positive number,
on average, the probability is on the magnitude of $(\frac{1}{2})^{N}:$%
\begin{eqnarray*}
E(\Pi _{m=1}^{N}\Pr (X_{m}+\xi _{m} &<&0))=E(\Pi _{m=1}^{N}(1-\Phi (X_{m}))
\\
&=&\Pi _{m=1}^{N}[1-E(\Phi (X_{m}))] \\
&=&(\frac{1}{2})^{N}
\end{eqnarray*}%
Therefore, even in the worst scenario that the chain effect $\delta $
approaches infinity, the probability of having a large connected area that
differs between $D^{U}$ and $D^{L}$ decreases exponentially with the size of
the area. In the current application, the size of the largest connected area
that differs between $D^{L}$ and $D^{U}$ is seldom bigger than seven or
eight markets.

\begin{thebibliography}{99}
\bibitem{} Ackerberg, D., and G. Gowrisankaran (2007): Quantifying
equilibrium network externalities in the ACH banking industry,\ forthcoming
RAND Journal of Economics.

\bibitem{} Andrews, D., S. Berry, and P. Jia (2004): Confidence Regions for
Parameters in Discrete Games with Multiple Equilibria, Unpublished
Manuscript, \ Yale University.

\bibitem{} Athey S. (2002): Monotone Comparative Statics\ Under Uncertainty,
Quarterly Journal of Economics, 117, 187-223.

\bibitem{} Bajari, Patrick, and Jeremy Fox (2005): Complementarities and
Collusion in an FCC Spectrum Auction, Unpublished Manuscript,\ University of
Chicago.

\bibitem{Bajari & Rox} Bajari, P., H. Hong, and S. Ryan (2004):
Identification and Estimation of Discrete Games of Complete Information,
Unpublished Manuscript, Duke University.

\bibitem{} Basker, E. \ (2005a): Job Creation or Destruction? Labor-Market
Effects of Wal-Mart Expansion,\ The Review of Economics and Statistics, 87,
174-183.

\bibitem{3} Basker,\ E. (2005b): Selling a Cheaper Mousetrap: Wal-Mart's
Effect on Retail Prices,\ Journal of Urban Economics, 58, 203-229.

\bibitem{} Berry, S. (1992): Estimation of a Model of Entry in the Airline
Industry,\ Econometrica, 60, 889-917.

\bibitem{} Bresnahan, T. and P. Reiss (1990): Entry into Monopoly Markets,\
Review of Economic Studies, 57, 531-553.

\bibitem{7} Bresnahan, T. and P. Reiss (1991): Entry and Competition in
Concentrated Markets,\ Journal of Political Economy, 95, 57-81.

\bibitem{} Ciliberto, F., and E. Tamer (2006): Market Structure and Multiple
Equilibria in Airline Markets, Unpublished Manuscript, Northwestern
University.

\bibitem{8} Chernozhukov, V., H. Hong, and E. Tamer (2007): Estimation and
Confidence Regions for Parameter Sets in Econometric Models, Econometrica,
75, 1243-1284.

\bibitem{} Committee on Small Business House (1994): Impact of Discount
Superstores on Small Business and Local Communities,\ Committee Serial No.
103-99. Congressional Information Services, Inc.

\bibitem{8b} Conley, T. (1999): GMM Estimation with Cross Sectional
Dependence,\ Journal of Econometrics, 92, 1-45.

\bibitem{} Conley, T., and E. Ligon (2002): Economic Distance and Cross
Country Spillovers, Journal of Economic Growth, 7, 157-187.

\bibitem{9} Davis, Peter. (2006): Spatial Competition in Retail Markets:
Movie Theaters,\ forthcoming RAND Journal of Economics.

\bibitem{9a} Directory of Discount Department Stores\ (1988-1997), Chain
Store Guide, Business Guides, Inc., New York.

\bibitem{9b} Discount Merchandiser\ (1988-1997), Schwartz Publications, New
York.

\bibitem{9c} Foster, L., J. Haltiwanger, and C.J. Krizan (2002): The Link
Between Aggregate and Micro Productivity Growth: Evidence from Retail
Trade,\ NBER working paper, No. 9120.

\bibitem{9d} Hausman, J., and E. Leibtag (2005): Consumer Benefits from
Increased Competition in Shopping Outlets: Measuring the Effect of
Wal-Mart,\ NBER working paper, No. 11809.

\bibitem{} Hess, S., and J. Polak (2003): A Comparison of Scrambled and
Shuffled Halton Sequences for Simulation Based Estimation, Unpublished
Manuscript, Imperial College London.

\bibitem{11b} Holmes, T. (2001): Barcodes Lead to Frequent Deliveries and
Superstores,\ RAND Journal of Economics, 32, 708-725.

\bibitem{11c} Holmes, T. (2005): The Diffusion of Wal-Mart and Economies of
Density, Unpublished Manuscript,\ University of Minnesota.

\bibitem{14} Kmart Inc. (1988-2000), Annual Report.

\bibitem{14a} Mazzeo, M. (2002): Product Choice and Oligopoly Market
Structure,\ RAND Journal of Economics, 33, 1-22.

\bibitem{14b} McFadden, D. (1989): A Method of Simulated Moments for
Estimation of Discrete Response Models without Numerical Integration,\
Econometrica, 57, 995-1026.

\bibitem{} Milgrom, P. and C. Shannon (1994): Monotone comparative statics,\
Econometrica, 62, 157-180.

\bibitem{newmark} Neumark, D., J. Zhang, and S. Ciccarella (2005): The
Effects of Wal-Mart on Local Labor Markets,\ NBER working paper, No. 11782.

\bibitem{} Newey, W. and K. West (1987): A Simple, Positive Semi-Definite,
Heteroskedasticity and Autocorrelation\ Consistent Covariance Matrix, 55,
703-708.

\bibitem{14c} Pakes, A. and D. Pollard: Simulation and the Asymptotics of
Optimization Estimators,\ Econometrica, 57, 1027-1057.

\bibitem{} Pakes, A., J. Porter, K. Ho, and J. Ishii (2005): Moment
Inequalities and Their Application, Unpublished Manuscript,\ Harvard
university.

\bibitem{14d} Pinkse, J., M. Slade, and C. Brett (2002): Spatial Price
Competition: A Semiparametric Approach,\ Econometrica, 70, 1111-1153.

\bibitem{} Romano J., and A. Shaikh (2006): Inference for the Identified Set
in Partially Identified Econometric Models, Unpublished Manuscript,\
University of Chicago.

\bibitem{15} Seim, K. (2006): An Empirical Model of Firm Entry with
Endogenous Product-Type Choices,\ RAND Journal of Economics, 37.

\bibitem{} Smith, H. (2004): Supermarket Choice and Supermarket Competition
in Market Equilibrium,\ Review of Economic Studies, 71, 235-263.

\bibitem{18} Stone, K. (1995): Impact of Wal-Mart Stores On Iowa
Communities: 1983-93,\ Economic Development Review, 13, 60-69.

\bibitem{} Tamer, E. (2003): Incomplete Simultaneous Discrete Response Model
with Multiple Equilibria,\ Review of Economic Studies, 70, 147-165.

\bibitem{} Tarski, A. (1955): A Lattice-Theoretical Fixpoint and Its
Applications,\ Pacific Journal of Mathematics, 5, 285-309.

\bibitem{18a} Taylor, D., and J. Archer (1994): Up Against the Wal-Marts
(How Your Business Can Survive in the Shadow of the Retail Giants),\ New
York: American Management Association.

\bibitem{} Topkis, D. (1978): Minimizing a submodular function on a
lattice,\ Operations Research, 26, 305-321.

\bibitem{18b} Topkis, D. (1998): Supermodularity and Complementarity,\
Princeton University Press, New Jersey.

\bibitem{18c} Train, K. (2000): Halton Sequences for Mixed Logit,
Unpublished Manuscript,\ UC Berkeley.

\bibitem{} Train, K. (2003): Discrete Choice Methods with Simulation,\
Cambridge University Press, Cambridge, UK.

\bibitem{18d} Vance, S., and R. Scott (1994): A History of Sam Walton's
Retail Phenomenon,\ Twayne Publishers, New York.

\bibitem{19} Wal-Mart Stores, Inc. (1970-2000), Annual Report.

\bibitem{} Zhou, L. (1994): The Set of Nash Equilibria of a Supermodular
Game Is a Complete Lattice,\ Games and Economic Behavior, 7, 295-300.

\bibitem{} Zhu, T., and V. Singh (2007): Spatial Competition with Endogenous
Location Choices: An Application to Discount Retailing, Unpublished
Manuscript,\ University of Chicago.
\end{thebibliography}

\end{document}
